\documentclass[12pt, titlepage]{article}

\usepackage{fullpage}
\usepackage[round]{natbib}
\usepackage{multirow}
\usepackage{booktabs}
\usepackage{tabularx}
\usepackage{graphicx}
\usepackage{float}
\usepackage{hyperref}
\hypersetup{
    colorlinks,
    citecolor=blue,
    filecolor=black,
    linkcolor=red,
    urlcolor=blue
}

\input{../../Comments}
%% Common Parts

\newcommand{\progname}{Chess Connect} % PUT YOUR PROGRAM NAME HERE
\newcommand{\authname}{Team \#4,
\\ Alexander Van Kralingen
\\ Arshdeep Aujla
\\ Jonathan Cels
\\ Joshua Chapman
\\ Rupinder Nagra} % AUTHOR NAMES without MacIDs 

\usepackage{hyperref}
    \hypersetup{colorlinks=true, linkcolor=blue, citecolor=blue, filecolor=blue,
                urlcolor=blue, unicode=false}
    \urlstyle{same}

\newcommand{\projectoverview}{

The Chess Connect project allows two users to play a game of chess on a physical board with the information being 
transmitted to an online web application over Bluetooth. Currently, there is no way for players to seamlessly switch 
between playing on a physical board and playing online, but Chess Connect intends to change this by creating a central 
platform that will provide flexibility and remove barriers for new players looking to learn the game.

}

\newcounter{acnum}
\newcommand{\actheacnum}{AC\theacnum}
\newcommand{\acref}[1]{AC\ref{#1}}

\newcounter{ucnum}
\newcommand{\uctheucnum}{UC\theucnum}
\newcommand{\uref}[1]{UC\ref{#1}}

\newcounter{mnum}
\newcommand{\mthemnum}{M\themnum}
\newcommand{\mref}[1]{M\ref{#1}}

\begin{document}

\title{Module Guide for \progname{}} 
\author{Author Name}
\date{\today}

\maketitle

\pagenumbering{roman}

\section{Revision History}

\begin{tabularx}{\textwidth}{p{3cm}p{2cm}X}
\toprule {\bf Date} & {\bf Version} & {\bf Notes}\\
\midrule
Date 1 & 1.0 & Notes\\
Date 2 & 1.1 & Notes\\
\bottomrule
\end{tabularx}

\newpage

\section{Reference Material}

This section records information for easy reference.

\subsection{Abbreviations and Acronyms}

\renewcommand{\arraystretch}{1.2}
\begin{tabular}{l l} 
  \toprule		
  \textbf{symbol} & \textbf{description}\\
  \midrule 
  AC & Anticipated Change\\
  DAG & Directed Acyclic Graph \\
  M & Module \\
  MG & Module Guide \\
  OS & Operating System \\
  R & Requirement\\
  SC & Scientific Computing \\
  SRS & Software Requirements Specification\\
  \progname & Explanation of program name\\
  UC & Unlikely Change \\
  \wss{etc.} & \wss{...}\\
  \bottomrule
\end{tabular}\\

\newpage

\tableofcontents

\listoftables

\listoffigures

\newpage

\pagenumbering{arabic}

\section{Introduction}

\subsection{Overview}

The Chess Connect project allows two users to play a game of chess on a physical board with the information being transmitted to an online web application over Bluetooth.
Currently, there is no way for players to seamlessly switch between playing on a physical board and playing online, but Chess Connect intends to change this by creating a central platform that will provide flexibility and remove barriers for new players looking to learn the game.

\subsection{Context}

This document is the Module Guide (MG), which is created after the Software Requirements Specification (SRS). 
The purpose of the Software Requirements Specification document is to present a description of the software system to be developed, including the functional and non-functional requirements for the project. The following MG has a different purpose, where it is instead providing a modular decomposition of the system, showing the modular structure of the application. The MG also describes how the requirements in the SRS are met with the modular structure that is described.
Along with the MG, it is also necessary to create a Module Interface Specification (MIS) explaining the semantics and syntax of each module. Examples of such semantics and syntax includes the access routines, state variables, inputs/outputs, and exceptions of the modules. This document will further expand on the information provided in the MG.

\subsection{Design Principles}

Information Hiding and Encapsulation are some of the design principles being used to build a modular structure of our application. The project should also assess the software metric of coupling and cohesion. Ideally, the project has high cohesion, and a low degree of coupling.
Information Hiding is the process of hiding the details of an object or function. This process disassociates the calling code from the internal workings of the object or function being called. This makes it possible to change the hidden portions without having to also change the calling code. Encapsulation is a design principle that makes it easier to enforce information hiding. Encapsulation simply hides the states of an object of a class by preventing direct access to it. 
A high cohesion signifies that the methods and data within a module are closely related. Low coupling means that there is a low degree of interdependence between the modules of the system. 


\subsection{Document Structure}
	The document structure is organised as follows:
\begin{itemize}

\item Section 4 lists Anticipated and Unlikely Changes to the system's implementation. This list is used for the Traceability Matrices later in the document.

\item Section 5 presents the Module Hierarchy, listing all the modules and their hierarchy by levels. 

\item Section 6 describes the Connection Between Requirements and Design, which details how the software requirements are related to the modules. 

\item Section 7 describes the Module Decomposition, detailing the module secrets, services, and implementations. 

\item Section 8 provides the Traceability Matrices. The first matrix connects the functional and nonfunctional requirements to the modules. The second matrix connects anticipated changes from Section 2 to the modules.

\item Section 9 presents the Uses Hierarchy diagram for the application. The model shows the uses relations between modules.

\end{itemize}	

\section{Anticipated and Unlikely Changes} \label{SecChange}

This section lists possible changes to the system. According to the likeliness
of the change, the possible changes are classified into two
categories. Anticipated changes are listed in Section \ref{SecAchange}, and
unlikely changes are listed in Section \ref{SecUchange}.

\subsection{Anticipated Changes} \label{SecAchange}

Anticipated changes are the source of the information that is to be hidden
inside the modules. Ideally, changing one of the anticipated changes will only
require changing the one module that hides the associated decision. The approach
adapted here is called design for
change.

\begin{description}
  \item[\refstepcounter{acnum} \actheacnum \label{acVerification}:] The format in which the data is passed from the hardware to the web application.
  \item[\refstepcounter{acnum} \actheacnum \label{acBoardState}:] The implementation of the data structure that stores the game board state.
  \item[\refstepcounter{acnum} \actheacnum \label{acFontAndPieces}:] User options to change the font and board size.
  \item[\refstepcounter{acnum} \actheacnum \label{acGUI}:] The GUI and interface design.
  \item[\refstepcounter{acnum} \actheacnum \label{acTimers}:] The amount of time for a move to be reflected on the web application.
\end{description}

\subsection{Unlikely Changes} \label{SecUchange}

\begin{description}
  \item[\refstepcounter{ucnum} \uctheucnum] The user option to play against another user online.
  \item[\refstepcounter{ucnum} \uctheucnum] The interface functionality of the system.
  \item[\refstepcounter{ucnum} \uctheucnum] The rules and values related to the game itself.
  \item[\refstepcounter{ucnum} \uctheucnum] The purpose of the system to allow users to play a game of chess against an opponent.
\end{description}

\section{Module Hierarchy} \label{SecMH}

This section provides an overview of the module design. Modules are summarized
in a hierarchy decomposed by secrets in Table \ref{TblMH}. The modules listed
below, which are leaves in the hierarchy tree, are the modules that will
actually be implemented.

\begin{description}
\item [\refstepcounter{mnum} \mthemnum \label{mHH}:] Hardware-Hiding Module
\item [\refstepcounter{mnum} \mthemnum \label{mInput}:] Web Application Input Module
\item [\refstepcounter{mnum} \mthemnum \label{mDisplay}:] Display Module
\item [\refstepcounter{mnum} \mthemnum \label{mOutput}:] Web Application Output Module
\item [\refstepcounter{mnum} \mthemnum \label{mMode}:] User Mode Module
\item [\refstepcounter{mnum} \mthemnum \label{mBoard}:] Board Module
\item [\refstepcounter{mnum} \mthemnum \label{mGame}:] Web Application Game State Module
\item [\refstepcounter{mnum} \mthemnum \label{mEngine}:] Engine Module
\end{description}


\begin{table}[h!]
\centering
\begin{tabular}{p{0.3\textwidth} p{0.6\textwidth}}
\toprule
\textbf{Level 1} & \textbf{Level 2}\\
\midrule

{Hardware-Hiding Module} & ~ \\
\midrule

\multirow{7}{0.3\textwidth}{Behaviour-Hiding Module} 
& Web Application Input Module\\
& Display Module\\
& Web Application Output Module\\
& ?\\
& ?\\
& ?\\ 
& ?\\
\midrule

\multirow{3}{0.3\textwidth}{Software Decision Module} 
& User Mode Module\\
& Board Module\\
& Web Application Game State Module\\
& Engine Module\\
& ?\\
& ?\\
& ?\\
\bottomrule

\end{tabular}
\caption{Module Hierarchy}
\label{TblMH}
\end{table}

\section{Connection Between Requirements and Design} \label{SecConnection}

The design satisfies the system requirements developed in the SRS. The Input module will handle all interactions from the inputs provided by the user, such as the FEN String, the user mode, and whether the game has terminated. The Display module outputs the visual component of the interface, and this module will cover all requirements that have a component of the user directly interacting with the system. The Board and Game modules cover the board setup and positions along with checking the end game conditions such as Checkmate and Stalemate. The look and feel requirements are satisfied by the input and output modules as they define the core areas of user interaction. The usability and humanity requirements are again satisfied by the input and output modules which define the user interaction. Performance requirements are to be satisfied by the combination of all modules, but the key modules that impact performance are software decision hiding modules. The operational and environmental requirements, and the maintainability and support requirements are achieved by all modules. This is because all modules are designed to be operational for the environment (the server and client) and maintained for a set time (end of semester). Security and legal requirements are covered by the entirety of the system. This is because the accessibility of the repository and the JavaScript standard in use affects all modules.

\section{Module Decomposition} \label{SecMD}

Modules are decomposed according to the principle of ``information hiding''
proposed by \citet{ParnasEtAl1984}. The \emph{Secrets} field in a module
decomposition is a brief statement of the design decision hidden by the
module. The \emph{Services} field specifies \emph{what} the module will do
without documenting \emph{how} to do it. For each module, a suggestion for the
implementing software is given under the \emph{Implemented By} title. If the
entry is \emph{OS}, this means that the module is provided by the operating
system or by standard programming language libraries.  \emph{\progname{}} means the
module will be implemented by the \progname{} software.

Only the leaf modules in the hierarchy have to be implemented. If a dash
(\emph{--}) is shown, this means that the module is not a leaf and will not have
to be implemented.

\subsection{Hardware Hiding Modules (\mref{mHH})}

\begin{description}
\item[Secrets:]The data structure and algorithm used to implement the virtual
  hardware.
\item[Services:]Serves as a virtual hardware used by the rest of the
  system. This module provides the interface between the hardware and the
  software. So, the system can use it to display outputs or to accept inputs.
\item[Implemented By:] OS
\end{description}

\subsection{Behaviour-Hiding Module}

\begin{description}
\item[Secrets:]The contents of the required behaviours.
\item[Services:]Includes programs that provide externally visible behaviour of
  the system as specified in the software requirements specification (SRS)
  documents. This module serves as a communication layer between the
  hardware-hiding module and the software decision module. The programs in this
  module will need to change if there are changes in the SRS.
\item[Implemented By:] --
\end{description}

    \subsubsection{Web Application Input Module (\mref{mInput})}
    \begin{description}
    \item[Secrets:] Input data.
    \item[Services:] Takes in input data of current board state to provide to other modules.
    \item[Implemented By:] \progname\ (Node.js libraries, Bluetooth)
    \end{description}
    
    \subsubsection{Display Module \mref{mDisplay}} 
    \begin{description}
    \item[Secrets:] Graphics output data.
    \item[Services:] Allows users to view the current board configuration of the system.
    \item[Implemented By:] \progname\ (React.js framework)
    \end{description}

    \subsubsection{Web Application Output Module \mref{mOutput}} 
    \begin{description}
    \item[Secrets:] None.
    \item[Services:] Takes game state and engine moves, encodes the data, and transmits it.
    \item[Implemented By:] \progname\ (React.js framework, Bluetooth)
    \end{description}

\subsubsection{Etc.}


\subsection{Software Decision Module}

\begin{description}
\item[Secrets:] The design decision based on mathematical theorems, physical
  facts, or programming considerations. The secrets of this module are
  \emph{not} described in the SRS.
\item[Services:] Includes data structure and algorithms used in the system that
  do not provide direct interaction with the user. 
  % Changes in these modules are more likely to be motivated by a desire to
  % improve performance than by externally imposed changes.
\item[Implemented By:] --
\end{description}

    \subsubsection{User Mode Module \mref{mMode}}
    \begin{description}
    \item[Secrets:] User mode logic and data.
    \item[Services:] Handles switching between user modes and communicating with mode-specific modules.
    \item[Implemented By:] \progname\ (React.js framework)
    \end{description}

    \subsubsection{Board Module \mref{mBoard}}
    \begin{description}
    \item[Secrets:] Board data.
    \item[Services:] Stores and modifies board state information.
    \item[Implemented By:] \progname\ (React.js framework)
    \end{description}

    \subsubsection{Web Application Game State Module \mref{mGame}}
    \begin{description}
    \item[Secrets:] Game state and data.
    \item[Services:] Handles checking the game state (none, check, checkmate, stalemate).
    \item[Implemented By:] \progname\ (React.js framework, Node.js libraries)
    \end{description}
    
    \subsubsection{Engine Module \mref{mEngine}}
    \begin{description}
    \item[Secrets:] Chess engine moves.
    \item[Services:] Uses the board state to calculate 3 possible moves.
    \item[Implemented By:] \progname\ (Node.js libraries and Stockfish)
    \end{description}

\section{Traceability Matrix} \label{SecTM}

This section shows two traceability matrices: between the modules and the
requirements and between the modules and the anticipated changes.

% the table should use mref, the requirements should be named, use something
% like fref
\begin{table}[H]
\centering
\begin{tabular}{p{0.2\textwidth} p{0.6\textwidth}}
\toprule
\textbf{Req.} & \textbf{Modules}\\
\midrule
R1 & \mref{mHH}, \mref{mInput}, \mref{mParams}, \mref{mControl}\\
R2 & \mref{mInput}, \mref{mParams}\\
R3 & \mref{mVerify}\\
R4 & \mref{mOutput}, \mref{mControl}\\
R5 & \mref{mOutput}, \mref{mODEs}, \mref{mControl}, \mref{mSeqDS}, \mref{mSolver}, \mref{mPlot}\\
R6 & \mref{mOutput}, \mref{mODEs}, \mref{mControl}, \mref{mSeqDS}, \mref{mSolver}, \mref{mPlot}\\
R7 & \mref{mOutput}, \mref{mEnergy}, \mref{mControl}, \mref{mSeqDS}, \mref{mPlot}\\
R8 & \mref{mOutput}, \mref{mEnergy}, \mref{mControl}, \mref{mSeqDS}, \mref{mPlot}\\
R9 & \mref{mVerifyOut}\\
R10 & \mref{mOutput}, \mref{mODEs}, \mref{mControl}\\
R11 & \mref{mOutput}, \mref{mODEs}, \mref{mEnergy}, \mref{mControl}\\
\bottomrule
\end{tabular}
\caption{Trace Between Requirements and Modules}
\label{TblRT}
\end{table}

\begin{table}[H]
\centering
\begin{tabular}{p{0.2\textwidth} p{0.6\textwidth}}
\toprule
\textbf{AC} & \textbf{Modules}\\
\midrule
\acref{acHardware} & \mref{mHH}\\
\acref{acInput} & \mref{mInput}\\
\acref{acParams} & \mref{mParams}\\
\acref{acVerify} & \mref{mVerify}\\
\acref{acOutput} & \mref{mOutput}\\
\acref{acVerifyOut} & \mref{mVerifyOut}\\
\acref{acODEs} & \mref{mODEs}\\
\acref{acEnergy} & \mref{mEnergy}\\
\acref{acControl} & \mref{mControl}\\
\acref{acSeqDS} & \mref{mSeqDS}\\
\acref{acSolver} & \mref{mSolver}\\
\acref{acPlot} & \mref{mPlot}\\
\bottomrule
\end{tabular}
\caption{Trace Between Anticipated Changes and Modules}
\label{TblACT}
\end{table}

\section{Use Hierarchy Between Modules} \label{SecUse}

In this section, the uses hierarchy between modules is
provided. \citet{Parnas1978} said of two programs A and B that A {\em uses} B if
correct execution of B may be necessary for A to complete the task described in
its specification. That is, A {\em uses} B if there exist situations in which
the correct functioning of A depends upon the availability of a correct
implementation of B.  Figure \ref{FigUH} illustrates the use relation between
the modules. It can be seen that the graph is a directed acyclic graph
(DAG). Each level of the hierarchy offers a testable and usable subset of the
system, and modules in the higher level of the hierarchy are essentially simpler
because they use modules from the lower levels.

\begin{figure}[H]
\centering
%\includegraphics[width=0.7\textwidth]{UsesHierarchy.png}
\caption{Use hierarchy among modules}
\label{FigUH}
\end{figure}

%\section*{References}

\bibliographystyle {plainnat}
\bibliography{../../../refs/References}

\end{document}