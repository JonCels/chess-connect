\documentclass[12pt, titlepage]{article}

\usepackage{amsmath, mathtools}

\usepackage[round]{natbib}
\usepackage{amsfonts}
\usepackage{amssymb}
\usepackage{graphicx}
\usepackage{colortbl}
\usepackage{xr}
\usepackage{hyperref}
\usepackage{longtable}
\usepackage{xfrac}
\usepackage{tabularx}
\usepackage{float}
\usepackage{siunitx}
\usepackage{booktabs}
\usepackage{multirow}
\usepackage[section]{placeins}
\usepackage{caption}
\usepackage{fullpage}

\hypersetup{
bookmarks=true,     % show bookmarks bar?
colorlinks=true,       % false: boxed links; true: colored links
linkcolor=red,          % color of internal links (change box color with linkbordercolor)
citecolor=blue,      % color of links to bibliography
filecolor=magenta,  % color of file links
urlcolor=cyan          % color of external links
}

\usepackage{array}

\externaldocument{../../SRS/SRS}

\input{../../Comments}
%% Common Parts

\newcommand{\progname}{Chess Connect} % PUT YOUR PROGRAM NAME HERE
\newcommand{\authname}{Team \#4,
\\ Alexander Van Kralingen
\\ Arshdeep Aujla
\\ Jonathan Cels
\\ Joshua Chapman
\\ Rupinder Nagra} % AUTHOR NAMES without MacIDs 

\usepackage{hyperref}
    \hypersetup{colorlinks=true, linkcolor=blue, citecolor=blue, filecolor=blue,
                urlcolor=blue, unicode=false}
    \urlstyle{same}

\newcommand{\projectoverview}{

The Chess Connect project allows two users to play a game of chess on a physical board with the information being transmitted to an online web application over Bluetooth.
Currently, there is no way for players to seamlessly switch between playing on a physical board and playing online, but Chess Connect intends to change this by creating a central platform that will provide flexibility and remove barriers for new players looking to learn the game.

}

\begin{document}

\title{Module Interface Specification for \progname{}}

\author{Author Name}

\date{\today}

\maketitle

\pagenumbering{roman}

\section{Revision History}

\addcontentsline{toc}{section}{Table of Revisions}
\section*{Table of Revisions}
\begin{table}[hp]
\caption{Revision History} \label{TblRevisionHistory}
\begin{tabularx}{\textwidth}{llX}
\toprule
\textbf{Date} & \textbf{Developer(s)} & \textbf{Change}\\
\midrule
2023-01-16 & Jonathan Cels, Rupinder Nagra & Web Application Modules\\
2023-01-18 & Jonathan Cels, Rupinder Nagra & Finalized Web Application Modules\\
\bottomrule
\end{tabularx}
\end{table}

~\newpage

\section{Symbols, Abbreviations and Acronyms}

\renewcommand{\arraystretch}{1.2}
\begin{tabular}{l l} 
  \toprule		
  \textbf{symbol} & \textbf{description}\\
  \midrule 
  M & Module \\
  MIS & Module Interface Specification \\
  R & Requirement\\
  \cite{FEN} & Forsyth-Edwards Notation\\
  \bottomrule
\end{tabular}\\

\newpage

\tableofcontents

\newpage

\pagenumbering{arabic}

\section{Introduction}

The following document details the Module Interface Specifications for
\wss{Fill in your project name and description}

Complementary documents include the System Requirement Specifications
and Module Guide.  The full documentation and implementation can be
found at \url{...}.  \wss{provide the url for your repo}

\section{Notation}

\wss{You should describe your notation.  You can use what is below as
  a starting point.}

The structure of the MIS for modules comes from \citet{HoffmanAndStrooper1995},
with the addition that template modules have been adapted from
\cite{GhezziEtAl2003}.  The mathematical notation comes from Chapter 3 of
\citet{HoffmanAndStrooper1995}.  For instance, the symbol := is used for a
multiple assignment statement and conditional rules follow the form $(c_1
\Rightarrow r_1 | c_2 \Rightarrow r_2 | ... | c_n \Rightarrow r_n )$.

The following table summarizes the primitive data types used by \progname. 

\begin{center}
\renewcommand{\arraystretch}{1.2}
\noindent 
\begin{tabular}{l l p{7.5cm}} 
\toprule 
\textbf{Data Type} & \textbf{Notation} & \textbf{Description}\\ 
\midrule
character & char & a single symbol or digit\\
integer & $\mathbb{Z}$ & a number without a fractional component in (-$\infty$, $\infty$) \\
natural number & $\mathbb{N}$ & a number without a fractional component in [1, $\infty$) \\
real & $\mathbb{R}$ & any number in (-$\infty$, $\infty$)\\
\bottomrule
\end{tabular} 
\end{center}

\noindent
The specification of \progname \ uses some derived data types: sequences, strings, and
tuples. Sequences are lists filled with elements of the same data type. Strings
are sequences of characters. Tuples contain a list of values, potentially of
different types. In addition, \progname \ uses functions, which
are defined by the data types of their inputs and outputs. Local functions are
described by giving their type signature followed by their specification.

\section{Module Decomposition}

The following table is taken directly from the Module Guide document for this project.

\begin{table}[h!]
\centering
\begin{tabular}{p{0.3\textwidth} p{0.6\textwidth}}
\toprule
\textbf{Level 1} & \textbf{Level 2}\\
\midrule

{Hardware-Hiding} & ~ \\
\midrule

\multirow{7}{0.3\textwidth}{Behaviour-Hiding} & Input Parameters\\
& Output Format\\
& Output Verification\\
& Temperature ODEs\\
& Energy Equations\\ 
& Control Module\\
& Specification Parameters Module\\
\midrule

\multirow{3}{0.3\textwidth}{Software Decision} & {Sequence Data Structure}\\
& ODE Solver\\
& Plotting\\
\bottomrule

\end{tabular}
\caption{Module Hierarchy}
\label{TblMH}
\end{table}

\newpage
~\newpage

\section{MIS of Web Application Input Module} \label{mInput}

    \subsection{Module}
    Web Application Input

    \subsection{Uses}
    \hyperref[mBoard]{Board Module}\\
    \hyperref[mMode]{User Mode Module}

    \subsection{Syntax}
    \subsubsection{Exported Constants}

    \subsubsection{Exported Access Programs}
        \begin{center}
        \begin{tabular}{p{4.5cm} p{3cm} p{4cm} p{2.5cm}}
        \hline
        \textbf{Name} & \textbf{In} & \textbf{Out} & \textbf{Exceptions} \\
        \hline
        parseInput & string & seq of string & invalidInput \\
        \hline
        \end{tabular}
        \end{center}

    \subsection{Semantics}
    \subsubsection{State Variables}
    \textbf{inputString:} string \#String containing \cite{FEN} string, user mode, game 
    termination state, and delimiting characters

    \subsubsection{Environment Variables}
    N/A

    \subsubsection{Assumptions}
    N/A

    \subsubsection{Access Routine Semantics}
        \noindent parseInput():
        \begin{itemize}
        \item output: sequence of strings. The first is the FEN string, the second is 
            the user mode, the third is the game termination state.
        \item exception: invalidInput if any of validFen, validUserMode, 
            or validGameTermination return false.
        \end{itemize}

    \subsubsection{Local Functions}
        \begin{center}
        \begin{tabular}{p{4.5cm} p{3cm} p{4cm} p{2.5cm}}
        \hline
        \textbf{Name} & \textbf{In} & \textbf{Out} & \textbf{Exceptions} \\
        \hline
        validFen & string & boolean & \\
        \hline
        validUserMode & string & boolean & \\
        \hline
        validGameTermination & string & boolean & \\
        \end{tabular}
        \end{center}
    
        ~\newpage

\section{MIS of Display Module} \label{mDisplay}

    \subsection{Module}
    Display

    \subsection{Uses}
    \hyperref[mBoard]{Board Module}

    \subsection{Syntax}
    \subsubsection{Exported Constants}

    \subsubsection{Exported Access Programs}
        \begin{center}
        \begin{tabular}{p{5cm} p{3.5cm} p{3cm} p{2.5cm}}
        \hline
        \textbf{Name} & \textbf{In} & \textbf{Out} & \textbf{Exceptions} \\
        \hline
        drawSquare & string & & \\
        \hline
        drawBoard & seq of (seq of int) & & \\
        \hline
        displayGameTermination & int & & \\
        \hline
        setBackground & string & & \\
        \hline
        \end{tabular}
        \end{center}

    \subsection{Semantics}
    \subsubsection{State Variables}
    N/A

    \subsubsection{Environment Variables}
    N/A

    \subsubsection{Assumptions}
    N/A

    \subsubsection{Access Routine Semantics}
        \noindent drawSquare():
        \begin{itemize}
            \item output: Draw board square
            \item exception: none
        \end{itemize}

        \noindent drawBoard():
        \begin{itemize}
            \item transition: Uses drawSquare to display the game board
            \item exception: none
        \end{itemize}

        \noindent displayGameTermination():
        \begin{itemize}
            \item transition: Displays game termination state (checkmate, stalemate, etc.)
            \item exception: none
        \end{itemize}

        \noindent setBackground():
        \begin{itemize}
            \item transition: Sets the background colors of the display.
            \item exception: none
        \end{itemize}

    \subsubsection{Local Functions}
    N/A

    \newpage

\section{MIS of Web Application Output Module} \label{mOutput}

    \subsection{Module}
    Web Application Output

    \subsection{Uses}
    \hyperref[mEngine]{Engine Module}\\
    \hyperref[mGame]{Game State Module}

    \subsection{Syntax}
    \subsubsection{Exported Constants}

    \subsubsection{Exported Access Programs}
        \begin{center}
        \begin{tabular}{p{4.5cm} p{4cm} p{3cm} p{2.5cm}}
        \hline
        \textbf{Name} & \textbf{In} & \textbf{Out} & \textbf{Exceptions} \\
        \hline
        sendData & string & string & \\
        \hline
        \end{tabular}
        \end{center}

    \subsection{Semantics}
    \subsubsection{State Variables}
    N/A

    \subsubsection{Environment Variables}
    N/A

    \subsubsection{Assumptions}
    N/A

    \subsubsection{Access Routine Semantics}
        \noindent sendData(string):
        \begin{itemize}
            \item output: string \#Encodes game state (none, check, 
                checkmate, stalemate), and 3 engine-generated moves
            \item exception: none
        \end{itemize}

    \subsubsection{Local Functions}
    N/A

    \newpage

\section{MIS of User Mode Module} \label{mMode}

    \subsection{Module}
    User Mode

    \subsection{Uses}
    \hyperref[mEngine]{Engine Module}

    \subsection{Syntax}
    \subsubsection{Exported Constants}

    \subsubsection{Exported Access Programs}
        \begin{center}
        \begin{tabular}{p{4.5cm} p{4cm} p{3cm} p{2.5cm}}
        \hline
        \textbf{Name} & \textbf{In} & \textbf{Out} & \textbf{Exceptions} \\
        \hline
        getUserMode & & string & \\
        \hline
        setUserMode & string & & \\
        \hline
        \end{tabular}
        \end{center}

    \subsection{Semantics}
    \subsubsection{State Variables}
    userMode: string \#Represents the current user mode (Normal, Beginner, Engine)

    \subsubsection{Environment Variables}
    N/A

    \subsubsection{Assumptions}
    N/A

    \subsubsection{Access Routine Semantics}
        \noindent getMode():
        \begin{itemize}
            \item output: string 
                \[output := userMode\]
            \item exception: none
        \end{itemize}

        \noindent setMode(string):
        \begin{itemize}
            \item transition: Sets userMode to the input user mode 
                \[userMode := input\]
            \item exception: none
        \end{itemize}

    \subsubsection{Local Functions}
    N/A

    \newpage

\section{MIS of Board Module} \label{mBoard}

    \subsection{Module}
    Board

    \subsection{Uses}
    \hyperref[mEngine]{Engine Module}\\
    \hyperref[mGame]{Game State Module}

    \subsection{Syntax}
    \subsubsection{Exported Constants}
        \#define letters [`a', `b', `c', `d', `e', `f', `g', `h']\\
        \#define startFEN = `rnbqkbnr/pppppppp/8/8/8/8/PPPPPPPP/RNBQKBNR w KQkq - 0 1'\\
        \#define boardDimension = 8

    \subsubsection{Exported Access Programs}
        \begin{center}
        \begin{tabular}{p{4.5cm} p{4cm} p{3cm} p{2.5cm}}
        \hline
        \textbf{Name} & \textbf{In} & \textbf{Out} & \textbf{Exceptions} \\
        \hline
        initialize & & & \\
        \hline
        getXYPosition & int & tuple of int & invalidIndex\\
        \hline
        getPosition & int & tuple of int & \\
        \hline
        getFenString & & string & \\
        \hline
        setFenString & string & & \\
        \hline
        \end{tabular}
        \end{center}

    \subsection{Semantics}
    \subsubsection{State Variables}
    fenString: string \#Stores FEN string of current game position

    \subsubsection{Environment Variables}
    N/A

    \subsubsection{Assumptions}
    initialize is called before any other access routine.

    \subsubsection{Access Routine Semantics}
        \noindent initialize():
        \begin{itemize}
            \item transition: \#Initializes fenString to the starting chess board position \\
                \[fenString := startFEN\]
            \item exception: none
        \end{itemize}

        \noindent getXYPosition(int: squareInd):
        \begin{itemize}
            \item output: \#X and Y number coordinate for an input square number. 
                Eg. getXYPosition(14) returns (0, 6). \\
                \[out := (squareInd\ //\ boardDimension,\ squareInd\ \%\ boardDimension)\]
            \item exception: none
        \end{itemize}

        \noindent getPosition(int: squareInd):
        \begin{itemize}
            \item output: \#letter and number coordinate for an input square number.
                Eg. getPosition(14) returns `g7'. \\
                \begin{multline}
                    out := `letters[squareInd\ \%\ boardDimension]' \notag\\ 
                    + `boardDimension - (squareInd\ //\ boardDimension)'
                \end{multline}
                
            \item exception: none
        \end{itemize}

        \noindent getFenString():
        \begin{itemize}
            \item output: \[out := fenString\]
            \item exception: none
        \end{itemize}

        \noindent setFenString(string: fen):
        \begin{itemize}
            \item transition: \[fenString := fen\]
            \item exception: none
        \end{itemize}

    \subsubsection{Local Functions}
    N/A

    \newpage

\section{MIS of Web Application Game State Module} \label{mGame}

    \subsection{Module}
    Web Application Game State

    \subsection{Uses}
    N/A

    \subsection{Syntax}
    \subsubsection{Exported Constants}

    \subsubsection{Exported Access Programs}
        \begin{center}
        \begin{tabular}{p{3.5cm} p{3cm} p{3cm} p{2.5cm}}
        \hline
        \textbf{Name} & \textbf{In} & \textbf{Out} & \textbf{Exceptions} \\
        \hline
        isCheck & string & boolean & \\
        \hline
        isCheckmate & string & boolean & \\
        \hline
        isStalemate & string & boolean & \\
        \hline
        \end{tabular}
        \end{center}

    \subsection{Semantics}
    \subsubsection{State Variables}
    N/A

    \subsubsection{Environment Variables}
    N/A

    \subsubsection{Assumptions}
    N/A

    \subsubsection{Access Routine Semantics}
        \noindent isCheck():
        \begin{itemize}
            \item output: True if the position is `check', false otherwise
            \item exception: none
        \end{itemize}

        \noindent isCheckmate():
        \begin{itemize}
            \item output: True if the position is `checkmate', false otherwise
            \item exception: none
        \end{itemize}

        \noindent isStalemate():
        \begin{itemize}
            \item output: True if the position is `stalemate', false otherwise
            \item exception: none
        \end{itemize}

    \subsubsection{Local Functions}
    N/A

\newpage

\section{MIS of Engine Module} \label{mEngine}
    \subsection{Module}
    Engine

    \subsection{Uses}
    N/A

    \subsection{Syntax}
    \subsubsection{Exported Constants}
    \#define depth \#How many layers of depth the chess engine should use to evaluate the position
    \#define maxSearchTime \#The maximum time the chess engine should take to evaluate the position
    
    \subsubsection{Exported Access Programs}
        \begin{center}
        \begin{tabular}{p{4cm} p{3cm} p{3cm} p{2.5cm}}
        \hline
        \textbf{Name} & \textbf{In} & \textbf{Out} & \textbf{Exceptions} \\
        \hline
        evaluatePosition & string & string & \\
        \hline
        \end{tabular}
        \end{center}

    \subsection{Semantics}
    \subsubsection{State Variables}
    N/A

    \subsubsection{Environment Variables}
    N/A

    \subsubsection{Assumptions}
    The depth and maxSearchTime values will determined experimentally after the system is built. 
    There is a trade-off between move quality and speed/depth of the search.

    \subsubsection{Access Routine Semantics}
        \noindent evaluatePosition(string):
        \begin{itemize}
            \item output: String containing 3 possible moves, calculated by a chess engine from the FEN input string
            \item exception: none
        \end{itemize}

    \subsubsection{Local Functions}
    N/A

\newpage

\bibliographystyle {plainnat}
\bibliography {../../../refs/References}

\newpage

\section{Appendix} \label{Appendix}

\wss{Extra information if required}

\end{document}