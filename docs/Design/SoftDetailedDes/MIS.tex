\documentclass[12pt, titlepage]{article}

\usepackage{amsmath, mathtools}

\usepackage[round]{natbib}
\usepackage{amsfonts}
\usepackage{amssymb}
\usepackage{graphicx}
\usepackage{colortbl}
\usepackage{xr}
\usepackage{hyperref}
\usepackage{longtable}
\usepackage{xfrac}
\usepackage{tabularx}
\usepackage{float}
\usepackage{siunitx}
\usepackage{booktabs}
\usepackage{multirow}
\usepackage[section]{placeins}
\usepackage{caption}
\usepackage{fullpage}
\usepackage{makecell}

\hypersetup{
bookmarks=true,     % show bookmarks bar?
colorlinks=true,       % false: boxed links; true: colored links
linkcolor=red,          % color of internal links (change box color with linkbordercolor)
citecolor=blue,      % color of links to bibliography
filecolor=magenta,  % color of file links
urlcolor=cyan          % color of external links
}

\usepackage{array}

\externaldocument{../../SRS/SRS}

\input{../../Comments}
%% Common Parts

\newcommand{\progname}{Chess Connect} % PUT YOUR PROGRAM NAME HERE
\newcommand{\authname}{Team \#4,
\\ Alexander Van Kralingen
\\ Arshdeep Aujla
\\ Jonathan Cels
\\ Joshua Chapman
\\ Rupinder Nagra} % AUTHOR NAMES without MacIDs 

\usepackage{hyperref}
    \hypersetup{colorlinks=true, linkcolor=blue, citecolor=blue, filecolor=blue,
                urlcolor=blue, unicode=false}
    \urlstyle{same}

\newcommand{\projectoverview}{

The Chess Connect project allows two users to play a game of chess on a physical board with the information being transmitted to an online web application over Bluetooth.
Currently, there is no way for players to seamlessly switch between playing on a physical board and playing online, but Chess Connect intends to change this by creating a central platform that will provide flexibility and remove barriers for new players looking to learn the game.

}

\begin{document}

\title{Module Interface Specification for \progname{}}

\author{\authname}

\date{\today}

\maketitle

\pagenumbering{roman}

\section{Revision History}

\begin{tabularx}{\textwidth}{p{3cm}p{2cm}X}
\toprule {\bf Date} & {\bf Version} & {\bf Notes}\\
\midrule
1/17/2023 & 1.0 & Notes\\
Date 2 & 1.1 & Notes\\
\bottomrule
\end{tabularx}

~\newpage

\section{Symbols, Abbreviations and Acronyms}

See SRS Documentation at \wss{give url}

\wss{Also add any additional symbols, abbreviations or acronyms}

\newpage

\tableofcontents

\newpage

\pagenumbering{arabic}

\section{Introduction}

The following document details the Module Interface Specifications for
\wss{Fill in your project name and description}

Complementary documents include the System Requirement Specifications
and Module Guide.  The full documentation and implementation can be
found at \url{...}.  \wss{provide the url for your repo}

\section{Notation}

\wss{You should describe your notation.  You can use what is below as
  a starting point.}

The structure of the MIS for modules comes from \citet{HoffmanAndStrooper1995},
with the addition that template modules have been adapted from
\cite{GhezziEtAl2003}.  The mathematical notation comes from Chapter 3 of
\citet{HoffmanAndStrooper1995}.  For instance, the symbol := is used for a
multiple assignment statement and conditional rules follow the form $(c_1
\Rightarrow r_1 | c_2 \Rightarrow r_2 | ... | c_n \Rightarrow r_n )$.

The following table summarizes the primitive data types used by \progname. 

\begin{center}
\renewcommand{\arraystretch}{1.2}
\noindent 
\begin{tabular}{l l p{7.5cm}} 
\toprule 
\textbf{Data Type} & \textbf{Notation} & \textbf{Description}\\ 
\midrule
character & char & a single symbol or digit\\
integer & $\mathbb{Z}$ & a number without a fractional component in (-$\infty$, $\infty$) \\
natural number & $\mathbb{N}$ & a number without a fractional component in [1, $\infty$) \\
real & $\mathbb{R}$ & any number in (-$\infty$, $\infty$)\\
boolean & bool & true (value of 1) or false (value of 0)\\
enumeration & enum & keywords assigned an integer value in order of declaration beginning at 0\\
structure & Piece & C++ struct data-type containing PieceType enumeration and int colour (0 for white, 1 for black)\\
\bottomrule
\end{tabular} 
\end{center}

\noindent
The specification of \progname \ uses some derived data types: sequences, strings, and
tuples. Sequences are lists filled with elements of the same data type. Strings
are sequences of characters. Tuples contain a list of values, potentially of
different types. In addition, \progname \ uses functions, which
are defined by the data types of their inputs and outputs. Local functions are
described by giving their type signature followed by their specification.

\section{Module Decomposition}

The following table is taken directly from the Module Guide document for this project.

\begin{table}[h!]
\centering
\begin{tabular}{p{0.3\textwidth} p{0.6\textwidth}}
\toprule
\textbf{Level 1} & \textbf{Level 2}\\
\midrule

{Hardware-Hiding} & ~ \\
\midrule

\multirow{7}{0.3\textwidth}{Behaviour-Hiding} & Input Parameters\\
& Output Format\\
& Output Verification\\
& Temperature ODEs\\
& Energy Equations\\ 
& Control Module\\
& Specification Parameters Module\\
\midrule

\multirow{3}{0.3\textwidth}{Software Decision} & {Sequence Data Structure}\\
& ODE Solver\\
& Plotting\\
\bottomrule

\end{tabular}
\caption{Module Hierarchy}
\label{TblMH}
\end{table}

\newpage
~\newpage


\section{MIS of Arduino Controller Module} \label{ArduinoControllerModule}

\subsection{Arduino Controller}

\subsection{Uses}
Arduino.h
SoftwareSerial.h
ChessBoard.h
PieceIdentification.h
Communication.h

\subsection{Syntax}

\subsubsection{Exported Constants}{
  None
}

\subsubsection{Exported Access Programs}
None

\subsection{Semantics}

\subsubsection{State Variables}{
  gameMode := enum \{ beginner, normal, engine \}\\
  gameState := enum \{ init, play, end, reset \}\\
  userAction := enum \{ wait\_white, wait\_black, piece\_lifted, remove\_piece, promoting, valid\_move, invalid\_move, draw, resign, reset \}\\
  boardState := \href{https://en.wikipedia.org/wiki/Forsyth%E2%80%93Edwards_Notation}{FEN string}
  playerWarning := enum \{ check, checkmate, stalemate \}\\
}

\subsubsection{Environment Variables}{
  HALL\_PINS: input pin addresses for receiving signal from Hall-effect sensors\\
  LED\_PINS: output pin addresses for lighting up the LEDs on the board\\
  rx\_from\_Teensy: input pin for communication with Teensy controller\\
  tx\_from\_Teensy: output pin for communication with Teensy controller\\
}

\subsubsection{Assumptions}{
  \begin{itemize}
    \item setup() will run before any other function.
    \item Connection exists between both controllers and remains constant
  \end{itemize}
  
}

\subsubsection{Access Routine Semantics}

\noindent setup():
\begin{itemize}
\item transition: initialize serial connection; read board state; game state set to "init"
\item exception: TeensyConnectionFailed
\end{itemize}

\noindent loop():
\begin{itemize}
\item transition: \begin{itemize}
  \item Main control loop.
  \item Polling sensors to update boardState FEN string.
  \item Checking for check/checkmate/stalemate signal from Web App to update playerWarning.
  \item Wait for userAction based on Hall-effect sensor inputs.
\end{itemize}
\item exception: TeensyConnectionFailed
\end{itemize}

\noindent changeGameState():
\begin{itemize}
\item transition: Change gameState based on user input button presses (game start, draw, reset).
\item exception: InvalidAction
\end{itemize}

\noindent changeGameMode():
\begin{itemize}
\item transition: Change gameMode based on user input button presses (beginner, normal, engine).
\item exception: InvalidAction
\end{itemize}

\noindent completeUserAction():
\begin{itemize}
\item transition: Update boardState based on completed userAction
\item exception: InvalidAction, UnknownAction
\end{itemize}

\noindent lightLED():
\begin{itemize}
\item output: LED\_pin := HIGH ($\mathbb{Z} := 1$) or LOW  ($\mathbb{Z} := 0$).
\item exception: TeensyConnectionFailed
\end{itemize}

\subsubsection{Local Functions}{
  None
}

\section{MIS of Piece Identification Module} \label{PieceIdentificationModule}

\subsection{Piece Identification}

\subsection{Uses}
Arduino.h

\subsection{Syntax}

\subsubsection{Exported Constants}{
  None
}

\subsubsection{Exported Access Programs}
None

\subsection{Semantics}

\subsubsection{State Variables}{
  None
}

\subsubsection{Environment Variables}{
  None
  
}

\subsubsection{Assumptions}{
  Hall-effect sensors will give accurate readings.
}

\subsubsection{Access Routine Semantics}

\noindent readSensors():
\begin{itemize}
\item output: Piece
\item exception: SensorOffline
\end{itemize}

\noindent waitForPiece():
\begin{itemize}
\item transition: Waiting to send signal based on a sensor transition from\\$HALL\_PIN[\mathbb{Z}][\mathbb{Z}] := \mathbb{R} \Rightarrow 0$
\item output: bool value of $(Piece Not Placed \Rightarrow false | Piece Placed \Rightarrow true)$
\item exception: SensorOffline
\end{itemize}

\wss{A module without environment variables or state variables is unlikely to
  have a state transition.  In this case a state transition can only occur if
  the module is changing the state of another module.}

\wss{Modules rarely have both a transition and an output.  In most cases you
  will have one or the other.}

\subsubsection{Local Functions}{
}

\wss{As appropriate} \wss{These functions are for the purpose of specification.
  They are not necessarily something that is going to be implemented
  explicitly.  Even if they are implemented, they are not exported; they only
  have local scope.}


\section{MIS of Chess Board Module} \label{ChessBoardModule} 

\subsection{Chess Board}

\wss{Short name for the module}

\subsection{Uses}
Arduino.h\\
PieceIdentification.h\\

\subsection{Syntax}

\subsubsection{Exported Constants}{
int numRows := Chess board rows \\
int numCols := Chess board columns \\
int LED\_PINS[numRows][numCols] := 2-D array controlling the LED output pins \\
int HALL\_PINS[numRows][numCols] := 2-D array controlling the Hall-effect sensor input pins \\
}

\subsubsection{Exported Access Programs}

\begin{center}
\begin{tabular}{p{2cm} p{4cm} p{4cm} p{2cm}}
\hline
\textbf{Name} & \textbf{In} & \textbf{Out} & \textbf{Exceptions} \\
\hline
movePiece & {int, int, int, int, PieceType} & boolean & InvalidMove \\
\midrule
removePiece & {int, int} & Piece & - \\
\midrule
isCheck & {int, int} & bool & - \\
\midrule
isCheckmate & {int, int} & bool & - \\
\midrule
boardToFEN & - & string & - \\
\midrule
sendFEN & string & - & - \\
\midrule
recieveMoves & - & Colour & - \\
\midrule
lightSquare & {int, int, Colour} & - & - \\
\hline
\end{tabular}
\end{center}

\subsection{Semantics}

\subsubsection{State Variables}{
  gameMode := enumeration\\
  check := boolean\\
  checkmate := boolean\\
  draw := boolean\\
}

\subsubsection{Environment Variables}{
  HALL\_PINS: input pins receiving signal from Hall-effect sensors\\
  LED\_PINS: output pins lighting up the LEDs on the board\\
  serialToTeensy: serial communication to and from the Teensy controller\\
}

\subsubsection{Assumptions}{
  \begin{itemize}
    \item Serial connection between both microcontrollers will remain constant
    \item All LED pins will remain connected
    \item Hall-effect sensors will function as intended
  \end{itemize}
}

\wss{Try to minimize assumptions and anticipate programmer errors via
  exceptions, but for practical purposes assumptions are sometimes appropriate.}

\subsubsection{Access Routine Semantics}

\noindent \wss{accessProg}():
\begin{itemize}
\item transition: \wss{if appropriate} 
\item output: \wss{if appropriate} 
\item exception: \wss{if appropriate} 
\end{itemize}

\wss{A module without environment variables or state variables is unlikely to
  have a state transition.  In this case a state transition can only occur if
  the module is changing the state of another module.}

\wss{Modules rarely have both a transition and an output.  In most cases you
  will have one or the other.}

\subsubsection{Local Functions}

pieceToChar: converting the piece type into the FEN-string character representation.

\wss{As appropriate} \wss{These functions are for the purpose of specification.
  They are not necessarily something that is going to be implemented
  explicitly.  Even if they are implemented, they are not exported; they only
  have local scope.}


\section{MIS of Communication Module} \label{CommunicationModule}

\subsection{Communication}

\subsection{Uses}
Arduino.h
SoftwareSerial.h

\subsection{Syntax}

\subsubsection{Exported Constants}{
  None
}

\subsubsection{Exported Access Programs}
None

\subsection{Semantics}

\subsubsection{State Variables}{
  command := The decoded message to update values (game state, game mode, light specific LED, etc.).\\
}

\subsubsection{Environment Variables}{
  messageEncoder := The string formatting to send a message to the Teensy Controller via Serial Communication.\\
  messageDecoder := The string formatting to read a message from the Teensy Controller via Serial Communication.\\
}

\subsubsection{Assumptions}{
  \begin{itemize}
    \item Communication string format remains consistent
    \item Connection exists between both controllers and remains constant
  \end{itemize}
  
}

\subsubsection{Access Routine Semantics}

\noindent encodeMessage():
\begin{itemize}
\item output: Translate game state or action into encoded string to be read by Teensy or the Web Application
\item exception: UnknownAction
\end{itemize}

\noindent decodeMessage():
\begin{itemize}
\item output: Translate encoded message from Teensy or the Web Application and convert into state change command
\item exception: UnknownCommand
\end{itemize}

\noindent processCommand():
\begin{itemize}
\item transition: Command received from Web Application or Teensy controller will be used to change the chess board accordingly.
\item --> This could be to change the game state, game mode, player warning (check, checkmate, stalemate) or to light appropriate LED's
\item exception: InvalidCommand
\end{itemize}

\subsubsection{Local Functions}{
  None
}



\newpage
\newpage

\bibliographystyle {plainnat}
\bibliography {../../../refs/References}

\newpage

\section{Appendix} \label{Appendix}

\wss{Extra information if required}

\end{document}