\documentclass[12pt, titlepage]{article}

\usepackage{amsmath, mathtools}

\usepackage[round]{natbib}
\usepackage{amsfonts}
\usepackage{amssymb}
\usepackage{graphicx}
\usepackage{colortbl}
\usepackage{xr}
\usepackage{hyperref}
\usepackage{longtable}
\usepackage{xfrac}
\usepackage{tabularx}
\usepackage{float}
\usepackage{siunitx}
\usepackage{booktabs}
\usepackage{multirow}
\usepackage[section]{placeins}
\usepackage{caption}
\usepackage{fullpage}
\usepackage{makecell}

\hypersetup{
bookmarks=true,     % show bookmarks bar?
colorlinks=true,       % false: boxed links; true: colored links
linkcolor=red,          % color of internal links (change box color with linkbordercolor)
citecolor=blue,      % color of links to bibliography
filecolor=magenta,  % color of file links
urlcolor=cyan          % color of external links
}

\usepackage{array}

\externaldocument{../../SRS/SRS}

\input{../../Comments}
%% Common Parts

\newcommand{\progname}{Chess Connect} % PUT YOUR PROGRAM NAME HERE
\newcommand{\authname}{Team \#4,
\\ Alexander Van Kralingen
\\ Arshdeep Aujla
\\ Jonathan Cels
\\ Joshua Chapman
\\ Rupinder Nagra} % AUTHOR NAMES without MacIDs 

\usepackage{hyperref}
    \hypersetup{colorlinks=true, linkcolor=blue, citecolor=blue, filecolor=blue,
                urlcolor=blue, unicode=false}
    \urlstyle{same}

\newcommand{\projectoverview}{

The Chess Connect project allows two users to play a game of chess on a physical board with the information being transmitted to an online web application over Bluetooth.
Currently, there is no way for players to seamlessly switch between playing on a physical board and playing online, but Chess Connect intends to change this by creating a central platform that will provide flexibility and remove barriers for new players looking to learn the game.

}

\begin{document}

\title{Module Interface Specification for \progname{}}

\author{Author Name}

\date{\today}

\maketitle

\pagenumbering{roman}

\section{Revision History}

\begin{tabularx}{\textwidth}{p{3cm}p{2cm}X}
\toprule {\bf Date} & {\bf Version} & {\bf Notes}\\
\midrule
1/17/2023 & 1.0 & Detailed Modules used by Arduino Mega 2560\\
Date 2 & 1.1 & Notes\\
\bottomrule
\end{tabularx}

~\newpage

\section{Symbols, Abbreviations and Acronyms}

See SRS Documentation at \wss{give url}

\wss{Also add any additional symbols, abbreviations or acronyms}

\newpage

\tableofcontents

\newpage

\pagenumbering{arabic}

\section{Introduction}

The following document details the Module Interface Specifications for
\wss{Fill in your project name and description}

Complementary documents include the System Requirement Specifications
and Module Guide.  The full documentation and implementation can be
found at \url{...}.  \wss{provide the url for your repo}

\section{Notation}

\wss{You should describe your notation.  You can use what is below as
  a starting point.}

The structure of the MIS for modules comes from \citet{HoffmanAndStrooper1995},
with the addition that template modules have been adapted from
\cite{GhezziEtAl2003}.  The mathematical notation comes from Chapter 3 of
\citet{HoffmanAndStrooper1995}.  For instance, the symbol := is used for a
multiple assignment statement and conditional rules follow the form $(c_1
\Rightarrow r_1 | c_2 \Rightarrow r_2 | ... | c_n \Rightarrow r_n )$.

The following table summarizes the primitive data types used by \progname. 

\begin{center}
\renewcommand{\arraystretch}{1.2}
\noindent 
\begin{tabular}{l l p{7.5cm}} 
\toprule 
\textbf{Data Type} & \textbf{Notation} & \textbf{Description}\\ 
\midrule
character & char & a single symbol or digit\\
integer & $\mathbb{Z}$ & a number without a fractional component in (-$\infty$, $\infty$) \\
natural number & $\mathbb{N}$ & a number without a fractional component in [1, $\infty$) \\
real & $\mathbb{R}$ & any number in (-$\infty$, $\infty$)\\
boolean & bool & true (value of 1) or false (value of 0)\\
enumeration & enum & keywords assigned an integer value in order of declaration beginning at 0\\
structure & Piece & C++ struct data-type containing PieceType enumeration and int colour (0 for white, 1 for black)\\
\bottomrule
\end{tabular} 
\end{center}

\noindent
The specification of \progname \ uses some derived data types: sequences, strings, and
tuples. Sequences are lists filled with elements of the same data type. Strings
are sequences of characters. Tuples contain a list of values, potentially of
different types. In addition, \progname \ uses functions, which
are defined by the data types of their inputs and outputs. Local functions are
described by giving their type signature followed by their specification.

\section{Module Decomposition}

The following table is taken directly from the Module Guide document for this project.

\begin{table}[h!]
\centering
\begin{tabular}{p{0.3\textwidth} p{0.6\textwidth}}
\toprule
\textbf{Level 1} & \textbf{Level 2}\\
\midrule

{Hardware-Hiding} & ~ \\
\midrule

\multirow{7}{0.3\textwidth}{Behaviour-Hiding} & Input Parameters\\
& Output Format\\
& Output Verification\\
& Temperature ODEs\\
& Energy Equations\\ 
& Control Module\\
& Specification Parameters Module\\
\midrule

\multirow{3}{0.3\textwidth}{Software Decision} & {Sequence Data Structure}\\
& ODE Solver\\
& Plotting\\
\bottomrule

\end{tabular}
\caption{Module Hierarchy}
\label{TblMH}
\end{table}

\newpage
~\newpage


\section{MIS of Arduino Controller Module} \label{ArduinoControllerModule}

\subsection{Arduino Controller}

\subsection{Uses}
Arduino\\
Software Serial\\
Chess Board\\
Piece Identification\\
Communication\\

\subsection{Syntax}

\subsubsection{Exported Constants}{
  None
}

\subsubsection{Exported Access Programs}
None

\subsection{Semantics}
{
  \begin{center}
  \begin{tabular}{p{2cm} p{4cm} p{4cm} p{2cm}}
  \hline
  \textbf{Name} & \textbf{In} & \textbf{Out} & \textbf{Exceptions} \\
  \hline
  setup & - & - & TeensyConnectionFailed \\
  \midrule
  loop & - & - & TeensyConnectionFailed \\
  \midrule
  changeGameState & gameState & gameState & InvalidAction \\
  \midrule
  changeGameMode & gameMode & string & InvalidAction \\
  \midrule
  competeUserAction & string & userAction & InvalidAction, UnknownAction \\
  \midrule
  lightLED & {int, int} & int & - \\
  \hline
  \end{tabular}
  \end{center}
}

\subsubsection{State Variables}{
  gameMode := enum \{ beginner, normal, engine \}\\
  gameState := enum \{ init, play, end, reset \}\\
  userAction := enum \{ wait\_white, wait\_black, piece\_lifted, remove\_piece, promoting, valid\_move, invalid\_move, draw, resign, reset \}\\
  boardState := \href{https://en.wikipedia.org/wiki/Forsyth%E2%80%93Edwards_Notation}{FEN string}
  playerWarning := enum \{ check, checkmate, stalemate \}\\
}

\subsubsection{Environment Variables}{
  HALL\_PINS: input pin addresses for receiving signal from Hall-effect sensors\\
  LED\_PINS: output pin addresses for lighting up the LEDs on the board\\
  rx\_from\_Teensy: input pin for communication with Teensy controller\\
  tx\_from\_Teensy: output pin for communication with Teensy controller\\
}

\subsubsection{Assumptions}{
  \begin{itemize}
    \item setup() will run before any other function.
    \item Connection exists between both controllers and remains constant
  \end{itemize}
  
}

\subsubsection{Access Routine Semantics}

\noindent loop():
\begin{itemize}
\item transition: \begin{itemize}
  \item Main control loop.
  \item Polling sensors to update boardState FEN string.
  \item Checking for check/checkmate/stalemate signal from Web App to update playerWarning.
  \item Wait for userAction based on Hall-effect sensor inputs.
\end{itemize}
\item exception: TeensyConnectionFailed
\end{itemize}

\noindent changeGameState():
\begin{itemize}
\item transition: Change gameState based on user input button presses (game start, draw, reset).
\item exception: InvalidAction
\end{itemize}

\noindent changeGameMode():
\begin{itemize}
\item transition: Change gameMode based on user input button presses (beginner, normal, engine).
\item exception: InvalidAction
\end{itemize}

\noindent completeUserAction():
\begin{itemize}
\item transition: Update boardState based on completed userAction
\item exception: InvalidAction, UnknownAction
\end{itemize}

\noindent lightLED():
\begin{itemize}
\item output: LED\_pin := HIGH ($\mathbb{Z} := 1$) or LOW  ($\mathbb{Z} := 0$).
\item exception: TeensyConnectionFailed
\end{itemize}

\subsubsection{Local Functions}{
  \noindent setup():
  \begin{itemize}
  \item transition: initialize serial connection; read board state; game state set to "init"
  \item exception: TeensyConnectionFailed
  \end{itemize}
}

\newpage

\section{MIS of Piece Identification Module} \label{PieceIdentificationModule}

\subsection{Piece Identification}

\subsection{Uses}{
  None
}

\subsection{Syntax}

\subsubsection{Exported Constants}{
  None
}

\subsubsection{Exported Access Programs}{
  \begin{center}
  \begin{tabular}{p{2cm} p{4cm} p{4cm} p{2cm}}
  \hline
  \textbf{Name} & \textbf{In} & \textbf{Out} & \textbf{Exceptions} \\
  \hline
  readSensors & {int} & Piece & SensorOffline \\
  \midrule
  waitForPiece & {int, int, Piece} & bool & PieceMissingTimeout \\
  \hline
  \end{tabular}
  \end{center}
}

\subsection{Semantics}

\subsubsection{State Variables}{
  None
}

\subsubsection{Environment Variables}{
  sensorInput: readings from various hall-effect sensors
}

\subsubsection{Assumptions}{
  Hall-effect sensors will give accurate readings.
}

\subsubsection{Access Routine Semantics}

\noindent readSensors():
\begin{itemize}
\item output: Piece
\item exception: SensorOffline
\end{itemize}

\noindent waitForPiece():
\begin{itemize}
\item transition: Waiting to send signal based on a sensor transition from\\$HALL\_PIN[\mathbb{Z}][\mathbb{Z}] := \mathbb{R} \Rightarrow 0$
\item output: bool value of $(Piece Not Placed \Rightarrow false | Piece Placed \Rightarrow true)$
\item exception: PieceMissingTimeout
\end{itemize}

\subsubsection{Local Functions}{
  None
}

\newpage

\section{MIS of Chess Board Module} \label{ChessBoardModule} 

\subsection{Chess Board}

\subsection{Uses}
Arduino\\
Piece Identification\\

\subsubsection{Exported Constants}{
int numRows : Chess board rows \\
int numCols : Chess board columns \\
int LED\_PINS[numRows][numCols] : 2-D array controlling the LED output pins \\
int HALL\_PINS[numRows][numCols] : 2-D array controlling the Hall-effect sensor input pins \\
}

\subsubsection{Exported Access Programs}{
  None
}

\subsection{Semantics}{
  \begin{center}
  \begin{tabular}{p{2cm} p{4cm} p{4cm} p{2cm}}
  \hline
  \textbf{Name} & \textbf{In} & \textbf{Out} & \textbf{Exceptions} \\
  \hline
  movePiece & {int, int, int, int, PieceType} & boolean & InvalidMove \\
  \midrule
  removePiece & {int, int} & Piece & InvalidMove \\
  \midrule
  isCheckmateCheckOrStalemate & {int, int} & bool & - \\
  \midrule
  boardToFEN & - & string & - \\
  \midrule
  recieveMoves & - & Colour & InvalidMove \\
  \midrule
  lightSquare & {int, int, Colour} & - & DigitalWriteFailed \\
  \midrule
  pieceToChar & Piece & char & - \\
  \hline
  \end{tabular}
  \end{center}
}

\subsubsection{State Variables}{
  gameMode := enumeration\\
  check := boolean\\
  checkmate := boolean\\
  draw := boolean\\
}

\subsubsection{Environment Variables}{
  HALL\_PINS: input pins receiving signal from Hall-effect sensors\\
  LED\_PINS: output pins lighting up the LEDs on the board\\
  serialToTeensy: serial communication to and from the Teensy controller\\
}

\subsubsection{Assumptions}{
  \begin{itemize}
    \item Serial connection between both microcontrollers will remain constant
    \item All LED pins will remain connected
    \item Hall-effect sensors will function as intended
  \end{itemize}
}
\noindent movePiece():
\begin{itemize}
\item transition: Update Piece type and colour on the "to" square, while removing the piece from the "from" square.
\item exception: InvalidMove
\end{itemize}

\noindent removePiece():
\begin{itemize}
\item transition: Update Piece type and colour on the "to" square, while removing the piece from the "from" square. Remove the piece taken by the opponent.
\item output: returns the Piece that was removed.
\item exception: InvalidMove
\end{itemize}

\noindent isCheckmateCheckOrStalemate():
\begin{itemize}
\item transition: Update game state based on a command sent from the Web Application.
\item exception: None
\end{itemize}

\noindent boardToFEN():
\begin{itemize}
\item output: FEN string representation of the current board state.
\item exception: None
\end{itemize}

\noindent recieveMoves():
\begin{itemize}
\item transition: Process best moves recieved from the web application and light appropriate LED's.
\item exception: InvalidMove
\end{itemize}

\noindent lightSquare():
\begin{itemize}
\item transition: Light appropriate LED's based on various conditions such as game mode, game state, check/mate/stalemate warning, etc.
\item exception: DigitalWriteFailed
\end{itemize}

\subsubsection{Local Functions}{
  \noindent pieceToChar():
  \begin{itemize}
  \item output: Converting the Piece type into the FEN-string character representation.
  \item exception: None
  \end{itemize}
}

\newpage


\section{MIS of Communication Module} \label{CommunicationModule}

\subsection{Communication}

\subsection{Uses}
Arduino.h
SoftwareSerial.h

\subsection{Syntax}

\subsubsection{Exported Constants}{
  None
}

\subsubsection{Exported Access Programs}{
  \begin{center}
  \begin{tabular}{p{2cm} p{4cm} p{4cm} p{2cm}}
  \hline
  \textbf{Name} & \textbf{In} & \textbf{Out} & \textbf{Exceptions} \\
  \hline
  encodeMessage & string & - & UnknownAction \\
  \midrule
  decodeMessage & - & string & UnknownCommand \\
  \midrule
  processCommand & string & string & InvalidCommand \\
  \hline
  \end{tabular}
  \end{center}
}

\subsection{Semantics}

\subsubsection{State Variables}{
  command: The decoded message to update values (game state, game mode, light specific LED, etc.).\\
}

\subsubsection{Environment Variables}{
  messageEncoder: The string formatting to send a message to the Teensy Controller via Serial Communication.\\
  messageDecoder: The string formatting to read a message from the Teensy Controller via Serial Communication.\\
}

\subsubsection{Assumptions}{
  \begin{itemize}
    \item Communication string format remains consistent
    \item Connection exists between both controllers and remains constant
  \end{itemize}
  
}

\subsubsection{Access Routine Semantics}

\noindent encodeMessage():
\begin{itemize}
\item output: Translate game state or action into encoded string to be read by Teensy or the Web Application
\item exception: UnknownAction
\end{itemize}

\noindent decodeMessage():
\begin{itemize}
\item output: Translate encoded message from Teensy or the Web Application and convert into state change command
\item exception: UnknownCommand
\end{itemize}

\noindent processCommand():
\begin{itemize}
\item transition: Command received from Web Application or Teensy controller will be used to change the chess board accordingly.
\item --> This could be to change the game state, game mode, player warning (check, checkmate, stalemate) or to light appropriate LED's
\item exception: InvalidCommand
\end{itemize}

\subsubsection{Local Functions}{
  None
}


\newpage

\section{MIS of Web Application Input Module} \label{mInput}

    \subsection{Module}
    Web Application Input

    \subsection{Uses}
    \hyperref[mBoard]{Board Module}\\
    \hyperref[mMode]{User Mode Module}

    \subsection{Syntax}
    \subsubsection{Exported Constants}
    
    \subsubsection{Exported Access Programs}
        \begin{center}
        \begin{tabular}{p{4.5cm} p{3cm} p{4cm} p{2.5cm}}
        \hline
        \textbf{Name} & \textbf{In} & \textbf{Out} & \textbf{Exceptions} \\
        \hline
        parseInput & string & seq of string & invalidInput \\
        \hline
        \end{tabular}
        \end{center}
    
    \subsection{Semantics}
    \subsubsection{State Variables}
    \textbf{inputString:} string \#String containing FEN string, user mode, game 
    termination state, and delimiting characters

    \subsubsection{Environment Variables}
        N/A

    \subsubsection{Assumptions}
        N/A
    
    \subsubsection{Access Routine Semantics}
        \noindent parseInput():
        \begin{itemize}
        \item output: sequence of strings. The first is the FEN string, the second is 
            the user mode, the third is the game termination state.
        \item exception: invalidInput if any of validFen, validUserMode, 
            or validGameTermination return false.
        \end{itemize}

    \subsubsection{Local Functions}
        \begin{center}
        \begin{tabular}{p{4.5cm} p{3cm} p{4cm} p{2.5cm}}
        \hline
        \textbf{Name} & \textbf{In} & \textbf{Out} & \textbf{Exceptions} \\
        \hline
        validFen & string & boolean & \\
        \hline
        validUserMode & string & boolean & \\
        \hline
        validGameTermination & string & boolean & \\
        \end{tabular}
        \end{center}
    
        ~\newpage
    

\section{MIS of Display Module} \label{mDisplay}

    \subsection{Module}
    Display

    \subsection{Uses}
    \hyperref[mBoard]{Board Module}

    \subsection{Syntax}
    \subsubsection{Exported Constants}

    \subsubsection{Exported Access Programs}
        \begin{center}
        \begin{tabular}{p{5cm} p{3.5cm} p{3cm} p{2.5cm}}
        \hline
        \textbf{Name} & \textbf{In} & \textbf{Out} & \textbf{Exceptions} \\
        \hline
        drawSquare & string & & \\
        \hline
        drawBoard & seq of (seq of int) & & \\
        \hline
        displayGameTermination & int & & \\
        \hline
        setBackground & string & & \\
        \hline
        \end{tabular}
        \end{center}

    \subsection{Semantics}
    \subsubsection{State Variables}
    N/A

    \subsubsection{Environment Variables}
    N/A

    \subsubsection{Assumptions}
    N/A

    \subsubsection{Access Routine Semantics}
        \noindent drawSquare():
        \begin{itemize}
            \item output: Draw board square
            \item exception: none
        \end{itemize}

        \noindent drawBoard():
        \begin{itemize}
            \item transition: Uses drawSquare to display the game board
            \item exception: none
        \end{itemize}

        \noindent displayGameTermination():
        \begin{itemize}
            \item transition: Displays game termination state (checkmate, stalemate, etc.)
            \item exception: none
        \end{itemize}

        \noindent setBackground():
        \begin{itemize}
            \item transition: Sets the background colors of the display.
            \item exception: none
        \end{itemize}

    \subsubsection{Local Functions}
    N/A

    \newpage

\section{MIS of Web Application Output Module} \label{mOutput}

    \subsection{Module}
    Web Application Output

    \subsection{Uses}
    \hyperref[mEngine]{Engine Module}\\
    \hyperref[mGame]{Game State Module}

    \subsection{Syntax}
    \subsubsection{Exported Constants}

    \subsubsection{Exported Access Programs}
        \begin{center}
        \begin{tabular}{p{4.5cm} p{4cm} p{3cm} p{2.5cm}}
        \hline
        \textbf{Name} & \textbf{In} & \textbf{Out} & \textbf{Exceptions} \\
        \hline
        sendData & string & string & \\
        \hline
        \end{tabular}
        \end{center}

    \subsection{Semantics}
    \subsubsection{State Variables}
    N/A

    \subsubsection{Environment Variables}
    N/A

    \subsubsection{Assumptions}
    N/A

    \subsubsection{Access Routine Semantics}
        \noindent sendData(string):
        \begin{itemize}
            \item output: string \#Encodes game state (none, check, 
                checkmate, stalemate), and 3 engine-generated moves
            \item exception: none
        \end{itemize}

    \subsubsection{Local Functions}
    N/A

    \newpage

\section{MIS of User Mode Module} \label{mMode}

    \subsection{Module}
    User Mode

    \subsection{Uses}
    \hyperref[mEngine]{Engine Module}

    \subsection{Syntax}
    \subsubsection{Exported Constants}

    \subsubsection{Exported Access Programs}
        \begin{center}
        \begin{tabular}{p{4.5cm} p{4cm} p{3cm} p{2.5cm}}
        \hline
        \textbf{Name} & \textbf{In} & \textbf{Out} & \textbf{Exceptions} \\
        \hline
        getUserMode & & string & \\
        \hline
        setUserMode & string & & \\
        \hline
        \end{tabular}
        \end{center}

    \subsection{Semantics}
    \subsubsection{State Variables}
    userMode: string \#Represents the current user mode (Normal, Beginner, Engine)

    \subsubsection{Environment Variables}
    N/A

    \subsubsection{Assumptions}
    N/A

    \subsubsection{Access Routine Semantics}
        \noindent getMode():
        \begin{itemize}
            \item output: string 
                \[output := userMode\]
            \item exception: none
        \end{itemize}

        \noindent setMode(string):
        \begin{itemize}
            \item transition: Sets userMode to the input user mode 
                \[userMode := input\]
            \item exception: none
        \end{itemize}

    \subsubsection{Local Functions}
    N/A

    \newpage

\section{MIS of Board Module} \label{mBoard}

    \subsection{Module}
    Board

    \subsection{Uses}
    \hyperref[mEngine]{Engine Module}\\
    \hyperref[mGame]{Game State Module}

    \subsection{Syntax}
    \subsubsection{Exported Constants}
        \#define letters [`a', `b', `c', `d', `e', `f', `g', `h']\\
        \#define startFEN = `rnbqkbnr/pppppppp/8/8/8/8/PPPPPPPP/RNBQKBNR w KQkq - 0 1'\\
        \#define boardDimension = 8

    \subsubsection{Exported Access Programs}
        \begin{center}
        \begin{tabular}{p{4.5cm} p{4cm} p{3cm} p{2.5cm}}
        \hline
        \textbf{Name} & \textbf{In} & \textbf{Out} & \textbf{Exceptions} \\
        \hline
        initialize & & & \\
        \hline
        getXYPosition & int & tuple of int & invalidIndex\\
        \hline
        getPosition & int & tuple of int & \\
        \hline
        getFenString & & string & \\
        \hline
        setFenString & string & & \\
        \hline
        \end{tabular}
        \end{center}

    \subsection{Semantics}
    \subsubsection{State Variables}
    fenString: string \#Stores FEN string of current game position

    \subsubsection{Environment Variables}
    N/A

    \subsubsection{Assumptions}
    initialize is called before any other access routine.

    \subsubsection{Access Routine Semantics}
        \noindent initialize():
        \begin{itemize}
            \item transition: \#Initializes fenString to the starting chess board position \\
                \[fenString := startFEN\]
            \item exception: none
        \end{itemize}

        \noindent getXYPosition(int: squareInd):
        \begin{itemize}
            \item output: \#X and Y number coordinate for an input square number. 
                Eg. getXYPosition(14) returns (0, 6). \\
                \[out := (squareInd\ //\ boardDimension,\ squareInd\ \%\ boardDimension)\]
            \item exception: none
        \end{itemize}

        \noindent getPosition(int: squareInd):
        \begin{itemize}
            \item output: \#letter and number coordinate for an input square number.
                Eg. getPosition(14) returns `g7'. \\
                \begin{multline}
                    out := `letters[squareInd\ \%\ boardDimension]' \notag\\ 
                    + `boardDimension - (squareInd\ //\ boardDimension)'
                \end{multline}
                
            \item exception: none
        \end{itemize}

        \noindent getFenString():
        \begin{itemize}
            \item output: \[out := fenString\]
            \item exception: none
        \end{itemize}

        \noindent setFenString(string: fen):
        \begin{itemize}
            \item transition: \[fenString := fen\]
            \item exception: none
        \end{itemize}

    \subsubsection{Local Functions}
    N/A

    \newpage

\section{MIS of Web Application Game State Module} \label{mGame}

    \subsection{Module}
    Web Application Game State

    \subsection{Uses}
    N/A

    \subsection{Syntax}
    \subsubsection{Exported Constants}

    \subsubsection{Exported Access Programs}
        \begin{center}
        \begin{tabular}{p{3.5cm} p{3cm} p{3cm} p{2.5cm}}
        \hline
        \textbf{Name} & \textbf{In} & \textbf{Out} & \textbf{Exceptions} \\
        \hline
        isCheck & string & boolean & \\
        \hline
        isCheckmate & string & boolean & \\
        \hline
        isStalemate & string & boolean & \\
        \hline
        \end{tabular}
        \end{center}

    \subsection{Semantics}
    \subsubsection{State Variables}
    N/A

    \subsubsection{Environment Variables}
    N/A

    \subsubsection{Assumptions}
    N/A

    \subsubsection{Access Routine Semantics}
        \noindent isCheck():
        \begin{itemize}
            \item output: True if the position is `check', false otherwise
            \item exception: none
        \end{itemize}

        \noindent isCheckmate():
        \begin{itemize}
            \item output: True if the position is `checkmate', false otherwise
            \item exception: none
        \end{itemize}

        \noindent isStalemate():
        \begin{itemize}
            \item output: True if the position is `stalemate', false otherwise
            \item exception: none
        \end{itemize}

    \subsubsection{Local Functions}
    N/A

\newpage

\section{MIS of Engine Module} \label{mEngine}
    \subsection{Module}
    Engine

    \subsection{Uses}
    N/A

    \subsection{Syntax}
    \subsubsection{Exported Constants}
    \#define depth \#How many layers of depth the chess engine should use to evaluate the position
    \#define maxSearchTime \#The maximum time the chess engine should take to evaluate the position
    
    \subsubsection{Exported Access Programs}
        \begin{center}
        \begin{tabular}{p{4cm} p{3cm} p{3cm} p{2.5cm}}
        \hline
        \textbf{Name} & \textbf{In} & \textbf{Out} & \textbf{Exceptions} \\
        \hline
        evaluatePosition & string & string & \\
        \hline
        \end{tabular}
        \end{center}

    \subsection{Semantics}
    \subsubsection{State Variables}
    N/A

    \subsubsection{Environment Variables}
    N/A

    \subsubsection{Assumptions}
    The depth and maxSearchTime values will determined experimentally after the system is built. 
    There is a trade-off between move quality and speed/depth of the search.

    \subsubsection{Access Routine Semantics}
        \noindent evaluatePosition(string):
        \begin{itemize}
            \item output: String containing 3 possible moves, calculated by a chess engine from the FEN input string
            \item exception: none
        \end{itemize}

    \subsubsection{Local Functions}
    N/A

\newpage
\newpage

\bibliographystyle {plainnat}
\bibliography {../../../refs/References}

\newpage

\section{Appendix} \label{Appendix}

\wss{Extra information if required}

\end{document}