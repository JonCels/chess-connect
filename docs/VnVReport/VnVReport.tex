\documentclass[12pt, titlepage]{article}

\usepackage{fullpage}
\usepackage[round]{natbib}
\usepackage{multirow}
\usepackage{booktabs}
\usepackage{tabularx}
\usepackage{graphicx}
\usepackage{float}
\usepackage{hyperref}
\usepackage[shortlabels]{enumitem}
\usepackage{amsmath, mathtools}
\usepackage{amsfonts}
\usepackage{amssymb}
\usepackage{colortbl}
\usepackage{xr}
\usepackage{longtable}
\usepackage{xfrac}
\usepackage{siunitx}
\usepackage{caption}
\usepackage{pdflscape}
\usepackage{afterpage}
\usepackage{titlesec}
\usepackage{array}
\hypersetup{
    colorlinks,
    citecolor=blue,
    filecolor=black,
    linkcolor=red,
    urlcolor=blue
}
\usepackage{graphicx}
\graphicspath{ {./images/} }
\usepackage{caption}
\usepackage{enumitem}
\usepackage{xcolor}
\usepackage{listings}

\input{../Comments}
%% Common Parts

\newcommand{\progname}{Chess Connect} % PUT YOUR PROGRAM NAME HERE
\newcommand{\authname}{Team \#4,
\\ Alexander Van Kralingen
\\ Arshdeep Aujla
\\ Jonathan Cels
\\ Joshua Chapman
\\ Rupinder Nagra} % AUTHOR NAMES without MacIDs 

\usepackage{hyperref}
    \hypersetup{colorlinks=true, linkcolor=blue, citecolor=blue, filecolor=blue,
                urlcolor=blue, unicode=false}
    \urlstyle{same}

\newcommand{\projectoverview}{

The Chess Connect project allows two users to play a game of chess on a physical board with the information being 
transmitted to an online web application over Bluetooth. Currently, there is no way for players to seamlessly switch 
between playing on a physical board and playing online, but Chess Connect intends to change this by creating a central 
platform that will provide flexibility and remove barriers for new players looking to learn the game.

}

% https://tex.stackexchange.com/a/409782
\colorlet{mygray}{black!30}
\colorlet{mygreen}{green!60!blue}
\colorlet{mymauve}{red!60!blue}

\lstset{
  backgroundcolor=\color{gray!10}, 
  basicstyle=\small,
  columns=fullflexible,
  breakatwhitespace=false,
  breaklines=true,
  captionpos=b,
  commentstyle=\color{mygreen},
  extendedchars=true,
  frame=single,
  keepspaces=true,
  keywordstyle=\color{blue}, 
  language=c++, 
  numbers=none,
  numbersep=5pt,
  numberstyle=\tiny\color{blue},
  rulecolor=\color{mygray}, 
  showspaces=false,
  showtabs=false,
  stepnumber=5,
  stringstyle=\color{mymauve}, 
  tabsize=3,
  title=\lstname 
}



\begin{document}

\title{System Verification and Validation Report for \progname{}} 
\author{\authname}
\date{\today}
	
\maketitle

\pagenumbering{roman}

\section{Revision History}

\begin{tabularx}{\textwidth}{p{3cm}p{2cm}X}
\toprule {\bf Date} & {\bf Version} & {\bf Notes}\\
\midrule
2023-03-04 & Arshdeep Aujla & Added Template for Nonfunctional Requirements\\
2023-03-05 & Arshdeep Aujla & Added Table for functional requirements, traceability matrix\\
2023-03-07 & Jonathan Cels & Added some nonfunctional requirement test reports\\
\bottomrule
\end{tabularx}

~\newpage

\section{Symbols, Abbreviations and Acronyms}

\renewcommand{\arraystretch}{1.2}
\begin{tabular}{l l} 
  \toprule		
  \textbf{symbol} & \textbf{description}\\
  \midrule 
  T & Test\\
  \bottomrule
\end{tabular}\\

Refer to SRS Section 1 for an extensive list of used symbols, abbreviations, and acronyms.

\newpage

\tableofcontents

\listoftables %if appropriate

\listoffigures %if appropriate

\newpage

\pagenumbering{arabic}

This document ...

\section{Functional Requirements Evaluation}
Refer to the VnV Plan for descriptions of the tests derived to evaluate the functional requirements.
\subsection{Game Active State}

\begin{table}[H]
    \centering
        \setlength{\leftmargini}{0.4cm}
        \begin{tabular}{| >{\centering\arraybackslash}m{1cm} | 
            >{\centering\arraybackslash}m{2.5cm} | 
            >{\centering\arraybackslash}m{4cm} | 
            >{\centering\arraybackslash}m{3cm} |
            >{\centering\arraybackslash}m{3cm} |
            >{\centering\arraybackslash}m{1.5cm} |}
        \hline
        \rowcolor[gray]{0.9}
        Test & Input & Expected & Actual & Notes & Result\\
        \hline
        GA-1 & Draw/resign button pressed while game active. & System variable `gameInProgress' set to false. &  System variable configured correctly. &  & Pass \\
        \hline
        GA-2 & Start game button pressed while game active. & System variable `gameInProgress' remains true. &  System variable configured correctly. &  & Pass \\
        \hline
        GA-3 & User mode button pressed while game active. & System variable `currMode' changed to represent the selected user mode. &  User mode unchanged. & Design changed, user mode not switchable while a game is active. Test fails by design. & Fail \\
        \hline
        GA-4 & Start game button pressed while game inactive. & System variable `gameInProgress' set to true, `currFEN' variable is set to the starting FEN. &  System variables configured correctly. &  & Pass \\
        \hline
        GA-5 & Move made that results in stalemate or checkmate according to the rules of chess. & System variable `gameInProgress' set to false. &  System variable configured correctly. &  & Pass \\ 
        \hline
        \end{tabular}
    \caption{Active State Functional Requirements Results}
\end{table}

\subsection{Game Inactive State}

\begin{table}[H]
    \centering
        \setlength{\leftmargini}{0.4cm}
        \begin{tabular}{| >{\centering\arraybackslash}m{1cm} | 
            >{\centering\arraybackslash}m{2.5cm} | 
            >{\centering\arraybackslash}m{4cm} | 
            >{\centering\arraybackslash}m{3cm} |
            >{\centering\arraybackslash}m{3cm} |
            >{\centering\arraybackslash}m{1.5cm} |}
        \hline
        \rowcolor[gray]{0.9}
        Test & Input & Expected & Actual & Notes & Result\\
        \hline
        GI-1 &  &  &  &  & \\
        \hline
        GI-2 &  &  &  &  & \\
        \hline
        GI-3 &  &  &  &  & \\
        \hline
        GI-4 &  &  &  &  & \\
        \hline
        GI-5 &  &  &  &  & \\
        \hline
        \end{tabular}
    \caption{Inactive State Functional Requirements Results}
\end{table}

\subsection{Normal Mode}

\begin{table}[H]
    \centering
        \setlength{\leftmargini}{0.4cm}
        \begin{tabular}{| >{\centering\arraybackslash}m{1cm} | 
            >{\centering\arraybackslash}m{2.5cm} | 
            >{\centering\arraybackslash}m{4cm} | 
            >{\centering\arraybackslash}m{3cm} |
            >{\centering\arraybackslash}m{3cm} |
            >{\centering\arraybackslash}m{1.5cm} |}
        \hline
        \rowcolor[gray]{0.9}
        Test & Input & Expected & Actual & Notes & Result\\
        \hline
        GA-1 &  &  &  &  & \\
        \hline
        GA-2 &  &  &  &  & \\
        \hline
        GA-3 &  &  &  &  & \\
        \hline
        GA-4 &  &  &  &  & \\
        \hline
        GA-5 &  &  &  &  & \\
        \hline
        \end{tabular}
    \caption{Normal Mode Functional Requirements Results}
\end{table}

\subsection{Engine Mode}

\begin{table}[H]
    \centering
        \setlength{\leftmargini}{0.4cm}
        \begin{tabular}{| >{\centering\arraybackslash}m{1cm} | 
            >{\centering\arraybackslash}m{2.5cm} | 
            >{\centering\arraybackslash}m{4cm} | 
            >{\centering\arraybackslash}m{3cm} |
            >{\centering\arraybackslash}m{3cm} |
            >{\centering\arraybackslash}m{1.5cm} |}
        \hline
        \rowcolor[gray]{0.9}
        Test & Input & Expected & Actual & Notes & Result\\
        \hline
        GA-1 &  &  &  &  & \\
        \hline
        GA-2 &  &  &  &  & \\
        \hline
        GA-3 &  &  &  &  & \\
        \hline
        GA-4 &  &  &  &  & \\
        \hline
        GA-5 &  &  &  &  & \\
        \hline
        \end{tabular}
    \caption{Engine Mode Functional Requirements Results}
\end{table}

\subsection{Beginner Mode}

\begin{table}[H]
    \centering
        \setlength{\leftmargini}{0.4cm}
        \begin{tabular}{| >{\centering\arraybackslash}m{1cm} | 
            >{\centering\arraybackslash}m{2.5cm} | 
            >{\centering\arraybackslash}m{4cm} | 
            >{\centering\arraybackslash}m{3cm} |
            >{\centering\arraybackslash}m{3cm} |
            >{\centering\arraybackslash}m{1.5cm} |}
        \hline
        \rowcolor[gray]{0.9}
        Test & Input & Expected & Actual & Notes & Result\\
        \hline
        GA-1 &  &  &  &  & \\
        \hline
        GA-2 &  &  &  &  & \\
        \hline
        GA-3 &  &  &  &  & \\
        \hline
        GA-4 &  &  &  &  & \\
        \hline
        GA-5 &  &  &  &  & \\
        \hline
        \end{tabular}
    \caption{Beginner Mode Functional Requirements Results}
\end{table}

\section{Nonfunctional Requirements Evaluation}
Refer to the VnV Plan for descriptions of the tests derived to evaluate the non-functional requirements.

\subsection{Look and Feel}

\begin{table}[H]
  \centering
    \setlength{\leftmargini}{0cm}
    \begin{tabular}{| >{\centering\arraybackslash}m{1.5cm} | 
      >{\centering\arraybackslash}m{4cm} | 
      >{\centering\arraybackslash}m{4cm} | 
      >{\centering\arraybackslash}m{4cm} |}
    \hline
    \rowcolor[gray]{0.9}
    Test & Inputs & Expected Result & Actual Result\\
    \hline
    NFT-1 &  &  & \\
    \hline
    \end{tabular}
  \caption{Look and Feel Non-Functional Requirements Results}
\end{table}
		
\subsection{Usability and Humanity}

\begin{table}[H]
  \centering
    \setlength{\leftmargini}{0cm}
    \begin{tabular}{| >{\centering\arraybackslash}m{1.5cm} | 
      >{\centering\arraybackslash}m{4cm} | 
      >{\centering\arraybackslash}m{4cm} | 
      >{\centering\arraybackslash}m{4cm} |}
    \hline
    \rowcolor[gray]{0.9}
    Test & Inputs & Expected Result & Actual Result\\
    \hline
    NFT-2 &  &  & \\
    \hline
    NFT-3 &  &  & \\
    \hline
    \end{tabular}
  \caption{Usability and Humanity Non-Functional Requirements Results}
\end{table}

\subsection{Performance}

\begin{table}[H]
  \centering
    \setlength{\leftmargini}{0cm}
    \begin{tabular}{| >{\centering\arraybackslash}m{1.5cm} | 
      >{\centering\arraybackslash}m{4cm} | 
      >{\centering\arraybackslash}m{4cm} | 
      >{\centering\arraybackslash}m{4cm} |}
    \hline
    \rowcolor[gray]{0.9}
    Test & Inputs & Expected Result & Actual Result\\
    \hline
    NFT-4 &  &  & \\
    \hline
    NFT-5 &  &  & \\
    \hline
    NFT-6 &  &  & \\
    \hline
    NFT-7 &  &  & \\
    \hline
    \end{tabular}
  \caption{Performance Non-Functional Requirements Results}
\end{table}

\subsection{Health and Safety}

\begin{table}[H]
  \centering
    \setlength{\leftmargini}{0cm}
    \begin{tabular}{| >{\centering\arraybackslash}m{1.5cm} | 
      >{\centering\arraybackslash}m{4cm} | 
      >{\centering\arraybackslash}m{4cm} | 
      >{\centering\arraybackslash}m{4cm} |}
    \hline
    \rowcolor[gray]{0.9}
    Test & Inputs & Expected Result & Actual Result\\
    \hline
    NFT-8 &  &  & \\
    \hline
    \end{tabular}
  \caption{Health and Safety Non-Functional Requirements Results}
\end{table}

\subsection{Precision and Accuracy}

\begin{table}[H]
  \centering
    \setlength{\leftmargini}{0cm}
    \begin{tabular}{| >{\centering\arraybackslash}m{1.5cm} | 
      >{\centering\arraybackslash}m{4cm} | 
      >{\centering\arraybackslash}m{4cm} | 
      >{\centering\arraybackslash}m{4cm} |}
    \hline
    \rowcolor[gray]{0.9}
    Test & Inputs & Expected Result & Actual Result\\
    \hline
    NFT-9 &  &  & \\
    \hline
    \end{tabular}
  \caption{Precision and Accuracy Non-Functional Requirements Results}
\end{table}

\subsection{Capacity}

\begin{table}[H]
  \centering
    \setlength{\leftmargini}{0cm}
    \begin{tabular}{| >{\centering\arraybackslash}m{1.5cm} | 
      >{\centering\arraybackslash}m{4cm} | 
      >{\centering\arraybackslash}m{4cm} | 
      >{\centering\arraybackslash}m{4cm} |}
    \hline
    \rowcolor[gray]{0.9}
    Test & Inputs & Expected Result & Actual Result\\
    \hline
    NFT-10 &  &  & \\
    \hline
    \end{tabular}
  \caption{Capacity Non-Functional Requirements Results}
\end{table}

\subsection{Security}

\begin{table}[H]
  \centering
    \setlength{\leftmargini}{0cm}
    \begin{tabular}{| >{\centering\arraybackslash}m{1.5cm} | 
      >{\centering\arraybackslash}m{4cm} | 
      >{\centering\arraybackslash}m{4cm} | 
      >{\centering\arraybackslash}m{4cm} |}
    \hline
    \rowcolor[gray]{0.9}
    Test & Inputs & Expected Result & Actual Result\\
    \hline
    NFT-11 &  &  & \\
    \hline
    NFT-12 &  &  & \\
    \hline
    \end{tabular}
  \caption{Security Non-Functional Requirements Results}
\end{table}

\section{Unit Testing}

  Creating unit tests for the Embedded software required several of Arduino's built in functionality to be simulated. This included serial communication functions,
  pin setup (input or output), reading from and writing to pins, and time delays. Additionally, binary values neded to be setup to simulate a sequence of events 
  such as values recorded from a hall sensor, or LEDs turning on or off. All of this is handled in the 
  \href{../../test/EmbeddedTest/ArduinoTest/MockArduinoController.cpp}{MockArduinoController.cpp} file, which holds the SerialStream and PinSimulation classes,
  as well as several functions for interacting with the hardware.

  Rather than unit testing every function in normal operation, individual functions were tested to ensure correct outputs from simulated inputs. Integration 
  with the system was completed physically with the Arduino exectuing the program. An example of the test for hardware is given below. The rest of the tests follow
  a similar format of setting up the initial state, simulating an input and comparing the expected output.

  
\begin{lstlisting}
    void testReadPiece()
    {
        setupBoard(); 
        
        // Simulate picking a piece 
        organizedHallValues[0][1] = randHall(NO_COLOUR);  
        mapHallValuesToSensors(); 
        PinSim.reWritePin(hallRx[0]); 
        writeAdcRow(hallRx[0], rawHallValues[0]); 

        //No Piece, No colour 
        Square expectedSquare = Square(0,1);    

        // Checkpick() function inside Arduino's loop should catch this,
        // updating the pieces on the board 
        loopArduino();  

        // Make sure the state changes to PIECE_LIFTED ('l')
        check(assert_equal('l', gameState), __FUNCTION__, __LINE__);  

        // Make sure the square in the board array is updated successfully
        check(assert_equal(expectedSquare, currentBoard[0][1]), __FUNCTION__, __LINE__);  
    }
\end{lstlisting}

testHighlightPawnValidMoves
testHighlightKnightValidMoves
testHighlightBishopValidMoves
testHighlightRookValidMoves
testHighlightQueenValidMoves
testHighlightKingValidMoves

\section{Changes Due to Testing}

\section{Automated Testing}
		
\section{Trace to Requirements}

\begin{longtable}{| p{.20\textwidth} | p{.80\textwidth} |}
  \hline
  Test & Requirement\\
  \hline
  GA-1 & GA1\\
  \hline
  GA-2 & GA2\\
  \hline
  GA-3 & GA3\\
  \hline
  GA-4 & GA6\\
  \hline
  GA-5 & GA7\\
  \hline
  GI-1 & GI1\\
  \hline
  GI-2 & GI2\\
  \hline
  GI-3 & GI3\\
  \hline
  GI-4 & GI4\\
  \hline
  GI-5 & GI5, GI6\\
  \hline
  NB-1 & NB1\\
  \hline
  NB-2 & NB2\\
  \hline
  NB-3 & NB3\\
  \hline
  ND-1 & ND1\\
  \hline
  NA-1 & NA1, NA2\\
  \hline
  NA-2 & NA3\\
  \hline
  EB-1 & EB1\\
  \hline
  EB-2 & EB2\\
  \hline
  EB-3 & EB3\\
  \hline
  EB-4 & EB4\\
  \hline
  ED-1 & ED1\\
  \hline
  ED-2 & ED2\\
  \hline
  EA-1 & EA1, EA2\\
  \hline
  EA-2 & EA3, EA4, EA5\\
  \hline
  EA-3 & EA6\\
  \hline
  BB-1 & BB1\\
  \hline
  BB-2 & BB2\\
  \hline
  BB-3 & BB3\\
  \hline
  BB-4 & BB4\\
  \hline
  BB-5 & BB5\\
  \hline
  BD-1 & BD1\\
  \hline
  BA-1 & BA1\\
  \hline
  BA-2 & BA2\\
  \hline
  NFT1 & LF3\\
  \hline
  NFT2 & UH5\\
  \hline
  NFT3 & UH6\\
  \hline
  NFT4 & PR1\\
  \hline
  NFT5 & PR2\\
  \hline
  NFT6 & PR3\\
  \hline
  NFT7 & PR4\\
  \hline
  NFT8 & PR6\\
  \hline
  NFT9 & PR7\\
  \hline
  NFT10 & PR10\\
  \hline
  NFT11 & SR4\\
  \hline
  NFT12 & SR3\\
  \hline
\caption{Requirements Traceability Matrix}
\end{longtable}
		
\section{Trace to Modules}		

\section{Code Coverage Metrics}

\appendix
\section{Reflection Appendix}


\bibliographystyle{plainnat}

\nocite{*}\bibliography{../../refs/References}

\end{document}