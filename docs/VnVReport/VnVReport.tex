\documentclass[12pt, titlepage]{article}

\usepackage{booktabs}
\usepackage{tabularx}
\usepackage{hyperref}
\usepackage{longtable}

\hypersetup{
    colorlinks,
    citecolor=black,
    filecolor=black,
    linkcolor=red,
    urlcolor=blue
}
\usepackage[round]{natbib}

\input{../Comments}
%% Common Parts

\newcommand{\progname}{Chess Connect} % PUT YOUR PROGRAM NAME HERE
\newcommand{\authname}{Team \#4,
\\ Alexander Van Kralingen
\\ Arshdeep Aujla
\\ Jonathan Cels
\\ Joshua Chapman
\\ Rupinder Nagra} % AUTHOR NAMES without MacIDs 

\usepackage{hyperref}
    \hypersetup{colorlinks=true, linkcolor=blue, citecolor=blue, filecolor=blue,
                urlcolor=blue, unicode=false}
    \urlstyle{same}

\newcommand{\projectoverview}{

The Chess Connect project allows two users to play a game of chess on a physical board with the information being 
transmitted to an online web application over Bluetooth. Currently, there is no way for players to seamlessly switch 
between playing on a physical board and playing online, but Chess Connect intends to change this by creating a central 
platform that will provide flexibility and remove barriers for new players looking to learn the game.

}

\begin{document}

\title{System Verification and Validation Report for \progname{}} 
\author{\authname}
\date{\today}
	
\maketitle

\pagenumbering{roman}

\section{Revision History}

\begin{tabularx}{\textwidth}{p{3cm}p{2cm}X}
\toprule {\bf Date} & {\bf Version} & {\bf Notes}\\
\midrule
2021-03-05 & Arshdeep Aujla & Added Template for Nonfunctional Requirements\\
Date 2 & 1.1 & Notes\\
\bottomrule
\end{tabularx}

~\newpage

\section{Symbols, Abbreviations and Acronyms}

\renewcommand{\arraystretch}{1.2}
\begin{tabular}{l l} 
  \toprule		
  \textbf{symbol} & \textbf{description}\\
  \midrule 
  T & Test\\
  \bottomrule
\end{tabular}\\

\wss{symbols, abbreviations or acronyms -- you can reference the SRS tables if needed}

\newpage

\tableofcontents

\listoftables %if appropriate

\listoffigures %if appropriate

\newpage

\pagenumbering{arabic}

This document ...

\section{Functional Requirements Evaluation}

\section{Nonfunctional Requirements Evaluation}

\subsection{Usability}
		
\subsection{Performance}

\subsection{etc.}
	
\section{Comparison to Existing Implementation}	

This section will not be appropriate for every project.

\section{Unit Testing}

\section{Changes Due to Testing}

\section{Automated Testing}
		
\section{Trace to Requirements}

\begin{longtable}{| p{.20\textwidth} | p{.80\textwidth} |}
  \hline
  Test & Requirement\\
  \hline
  GA-1 & GA1\\
  \hline
  GA-2 & GA2\\
  \hline
  GA-3 & GA3\\
  \hline
  GA-4 & GA6\\
  \hline
  GA-5 & GA7\\
  \hline
  GI-1 & GI1\\
  \hline
  GI-2 & GI2\\
  \hline
  GI-3 & GI3\\
  \hline
  GI-4 & GI4\\
  \hline
  GI-5 & GI5, GI6\\
  \hline
  NB-1 & NB1\\
  \hline
  NB-2 & NB2\\
  \hline
  NB-3 & NB3\\
  \hline
  ND-1 & ND1\\
  \hline
  NA-1 & NA1, NA2\\
  \hline
  NA-2 & NA3\\
  \hline
  EB-1 & EB1\\
  \hline
  EB-2 & EB2\\
  \hline
  EB-3 & EB3\\
  \hline
  EB-4 & EB4\\
  \hline
  ED-1 & ED1\\
  \hline
  ED-2 & ED2\\
  \hline
  EA-1 & EA1, EA2\\
  \hline
  EA-2 & EA3, EA4, EA5\\
  \hline
  EA-3 & EA6\\
  \hline
  BB-1 & BB1\\
  \hline
  BB-2 & BB2\\
  \hline
  BB-3 & BB3\\
  \hline
  BB-4 & BB4\\
  \hline
  BB-5 & BB5\\
  \hline
  BD-1 & BD1\\
  \hline
  BA-1 & BA1\\
  \hline
  BA-2 & BA2\\
  \hline
  NFT1 & LF3\\
  \hline
  NFT2 & UH5\\
  \hline
  NFT3 & UH6\\
  \hline
  NFT4 & PR1\\
  \hline
  NFT5 & PR2\\
  \hline
  NFT6 & PR3\\
  \hline
  NFT7 & PR4\\
  \hline
  NFT8 & PR6\\
  \hline
  NFT9 & PR7\\
  \hline
  NFT10 & PR10\\
  \hline
  NFT11 & SR4\\
  \hline
  NFT12 & SR3\\
  \hline
\caption{Requirements Traceability Matrix}
\end{longtable}
		
\section{Trace to Modules}		

\section{Code Coverage Metrics}

\appendix
\section{Reflection Appendix}


\bibliographystyle{plainnat}

\bibliography{../../refs/References}

\end{document}