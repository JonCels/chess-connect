\documentclass[12pt]{article}

\usepackage{amsmath, mathtools}
\usepackage{amsfonts}
\usepackage{amssymb}
\usepackage{graphicx}
\usepackage{colortbl}
\usepackage{xr}
\usepackage{hyperref}
\usepackage{longtable}
\usepackage{xfrac}
\usepackage{tabularx}
\usepackage{float}
\usepackage{siunitx}
\usepackage{booktabs}
\usepackage{caption}
\usepackage{pdflscape}
\usepackage{afterpage}
\usepackage{fullpage}
\usepackage[round]{natbib}
\usepackage{array}

\input{../Comments}
%% Common Parts

\newcommand{\progname}{Chess Connect} % PUT YOUR PROGRAM NAME HERE
\newcommand{\authname}{Team \#4,
\\ Alexander Van Kralingen
\\ Arshdeep Aujla
\\ Jonathan Cels
\\ Joshua Chapman
\\ Rupinder Nagra} % AUTHOR NAMES without MacIDs 

\usepackage{hyperref}
    \hypersetup{colorlinks=true, linkcolor=blue, citecolor=blue, filecolor=blue,
                urlcolor=blue, unicode=false}
    \urlstyle{same}

\newcommand{\projectoverview}{

The Chess Connect project allows two users to play a game of chess on a physical board with the information being 
transmitted to an online web application over Bluetooth. Currently, there is no way for players to seamlessly switch 
between playing on a physical board and playing online, but Chess Connect intends to change this by creating a central 
platform that will provide flexibility and remove barriers for new players looking to learn the game.

}

\begin{document}

\title{Software Requirements Specification for Chess Connect: Online tools combined with on-board vision to improve and share your game} 
\author{\authname}
\date{October 4th, 2022}
	
\maketitle

~\newpage

\tableofcontents

~\newpage

\addcontentsline{toc}{section}{Table of Revisions}
\section*{Table of Revisions}
\begin{table}[hp]
\caption{Revision History} \label{TblRevisionHistory}
\begin{tabularx}{\textwidth}{llX}
\toprule
\textbf{Date} & \textbf{Developer(s)} & \textbf{Change}\\
\midrule
2022-10-04 & Jonathan Cels & Template creation and document formatting\\ 
date & name & change\\
\bottomrule
\end{tabularx}
\end{table}

~\newpage

\section{Units, Terms, Acronyms, and Abbreviations}

\subsection{Table of Units}
Throughout this document SI (Syst\`{e}me International d'Unit\'{e}s) is employed
as the unit system.  In addition to the basic units, several derived units are
used as described below.  For each unit, the symbol is given followed by a
description of the unit and the SI name.

\begin{table}[ht]
  \noindent \begin{tabular}{l l l} 
    \toprule		
    \textbf{symbol} & \textbf{unit} & \textbf{SI}\\
    \midrule 
    \si{\volt} & electric potential & volt\\
    \si{\ampere} & current	& ampere\\
    \si{\ohm} & resistance	& ohm\\
    \si{\second} & time & second\\
    \si{\celsius} & temperature & centigrade\\
    \si{\joule} & energy & joule\\
    \si{\watt} & power & watt (W = \si{\joule\per\second})\\
    \bottomrule
  \end{tabular}
\end{table}

\newpage

\subsection{Abbreviations and Acronyms}
\begin{tabular}{l l} 
  \toprule		
  \textbf{symbol} & \textbf{description}\\
  \midrule 
  A & Assumption\\
  DD & Data Definition\\
  GD & General Definition\\
  GS & Goal Statement\\
  IM & Instance Model\\
  LC & Likely Change\\
  LCD & Liquid Crystal Display\\
  LED & Light-Emmitting Diode\\
  MCU & Micro Controller Unit\\
  PS & Physical System Description\\
  R & Requirement\\
  SRS & Software Requirements Specification\\
  T & Theoretical Model\\
  \bottomrule
\end{tabular}\\

\subsection{Mathematical Notation}

\subsection{Terminology and  Definitions}

\section{Introduction}
\subsection{Document Purpose}
\subsection{Characteristics of Intended Reader}
\subsection{Characteristics of Intended User}
\subsection{Stakeholders}

\section{Problem Description}

\section{Assumptions}

\section{Constraints}

\section{Scope}

\section{Project Overview}
\subsection{System Context Diagram}
\subsection{Normal Operation}
\subsubsection{Description}
\subsubsection{Use Cases/Scenarios}

\subsection{Behaviour Overview}
\subsection{Undesired Scenario Handling}

\section{System Level Variables}
\subsection{Constants}

\begin{table}[H]
  \centering
      \setlength{\leftmargini}{0.4cm}
      \begin{tabular}{| >{\centering\arraybackslash}m{2cm} | 
        >{\centering\arraybackslash}m{2cm} | 
        >{\centering\arraybackslash}m{10cm} |}
      \hline
      \rowcolor[gray]{0.9}
      Constant & Unit & Value\\
      \hline
      Chess board width & inches & 12\\
     \hline
     Chess board length & inches & 12\\
     \hline
     Chess board tile width & inches & 1.5\\
     \hline 
     Chess board tile length & inches & 1.5\\ 
     \hline 
     Supply Power to Board & Voltage & 110 VAC\\
      \end{tabular}
  \label{Table}
  \end{table}

\subsection{Monitored Variables}

\begin{table}[H]
  \centering
      \setlength{\leftmargini}{0.4cm}
      \begin{tabular}{| >{\centering\arraybackslash}m{2.5cm} | 
        >{\centering\arraybackslash}m{2cm} | 
        >{\centering\arraybackslash}m{9cm} |}
      \hline
      \rowcolor[gray]{0.9}
      Variable & Units & Description\\
      \hline
      s\_a\{1-8\} & Voltage & States of tiles a1 - a8 on the board. They are analog signals 
      converted to digital and the state of the tile is determined. The possible states of 
      each tile is empty, black/white pawn, black/white rook, black/white knight, 
      black/white bishop, black/white queen, black/white king. \\
      \hline
      s\_b\{1-8\} & Voltage & States of tiles b1 - b8 on the board. " " \\
      \hline
      s\_c\{1-8\} & Voltage & States git of tiles c1 - c8 on the board. " " \\
      \hline
      s\_d\{1-8\} & Voltage & States of tiles d1 - d8 on the board. " " \\
      \hline
      s\_e\{1-8\} & Voltage & States of tiles e1 - e8 on the board. " " \\
      \hline
      s\_f\{1-8\} & Voltage & States of tiles f1 - f8 on the board. " " \\
      \hline
      s\_g\{1-8\} & Voltage & States of tiles g1 - g8 on the board. " " \\
      \hline
      sw\_3pos\_p1 & Voltage & The three-position switch for player 1 is located on 
      top of the board. It toggles between the beginner advice, engine advice and no
      advice modes.\\
      \hline
      engine\_move & chess notation & The chess engine API provides best moves into 
      the system. \\
      \hline 
      \end{tabular}
  \label{Table}
  \end{table}

\subsection{Controlled Variables}

\begin{table}[H]
  \centering
      \setlength{\leftmargini}{0.4cm}
      \begin{tabular}{| >{\centering\arraybackslash}m{3cm} | 
        >{\centering\arraybackslash}m{2cm} | 
        >{\centering\arraybackslash}m{9cm} |}
      \hline
      \rowcolor[gray]{0.9}
      Variable & Units & Description\\
      \hline 
      LED\_row\{1-9\} & Voltage & A total of 81 LEDS will be located under the board. They 
      are on the corner of each tile and illuminate based on conditions of the inputs. \\
      \hline 
      LCD\_Display & Voltage & An LCD Display is located on the chess board to indicate
      best moves delivered by the engine. \\
      \hline 
      \end{tabular}
  \label{Table}
  \end{table}

\section{Requirements}
\subsection{Functional Requirements}
\subsection{Nonfunctional Requirements}

\section{Likely Changes}
\section{Unlikely Changes}

\section{Traceability Matrix}

\appendix
\section{Values of Auxiliary Constants}

\newpage

\appendix
\section{Reflection}
\subsection{Skills for Success}
\subsection{Knowledge and Learning Approaches}
\end{document}