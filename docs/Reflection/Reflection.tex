\documentclass{article}

\usepackage{tabularx}
\usepackage{booktabs}

\title{Reflection Report on \progname}

\author{\authname}

\date{}

\input{../Comments}
%% Common Parts

\newcommand{\progname}{Chess Connect} % PUT YOUR PROGRAM NAME HERE
\newcommand{\authname}{Team \#4,
\\ Alexander Van Kralingen
\\ Arshdeep Aujla
\\ Jonathan Cels
\\ Joshua Chapman
\\ Rupinder Nagra} % AUTHOR NAMES without MacIDs 

\usepackage{hyperref}
    \hypersetup{colorlinks=true, linkcolor=blue, citecolor=blue, filecolor=blue,
                urlcolor=blue, unicode=false}
    \urlstyle{same}

\newcommand{\projectoverview}{

The Chess Connect project allows two users to play a game of chess on a physical board with the information being transmitted to an online web application over Bluetooth.
Currently, there is no way for players to seamlessly switch between playing on a physical board and playing online, but Chess Connect intends to change this by creating a central platform that will provide flexibility and remove barriers for new players looking to learn the game.

}

\begin{document}

\begin{table}[hp]
\caption{Revision History} \label{TblRevisionHistory}
\begin{tabularx}{\textwidth}{llX}
\toprule
\textbf{Date} & \textbf{Developer(s)} & \textbf{Change}\\
\midrule
April 5, 2023 & Rupinder Nagra & Wrote project overview and Key Accomplishments\\
April 5, 2023 & Jonathan Cels & Wrote Key Problem Areas\\
April 5, 2023 & Joshua Chapman & What Would you Do Differently Next Time\\
April 5, 2023 & Arshdeep Aujla & Wrote section on limitations, expanded on section 2 and 3\\
\bottomrule
\end{tabularx}
\end{table}

\newpage

\maketitle

In this reflection report, we delve into the challenges and successes encountered while developing a solution aimed at bridging the divide between over-the-board and online chess experiences. We try to understand the critical factors that influenced the outcome and identify strategies for refining our approach to ensure the success in future endeavors.

\section{Project Overview}
\projectoverview

\section{Key Accomplishments}

Integrating the best of both worlds, the purpose of our project was to bridge the gap between over-the-board and online chess experiences, while overcoming challenges and celebrating our progress along the way.

\subsection{Successful initial hardware testing}
The fact that the hardware was functional for a short period indicates that the team was able to assemble and integrate the components successfully. This demonstrates an understanding of the hardware requirements and shows progress in the development process.

\subsection{Fully functional software}
The software being completely functional is a significant accomplishment. It means that the coding, design, and implementation of the software are efficient and effective. This is a critical component in creating a seamless experience for users who want to integrate their over-the-board and online play.

\subsection{Identification of potential issues}
Although the hardware did not function during the live demo, this failure provides valuable information on potential problems that need to be addressed. The team can now investigate the issue, identify the root cause, and implement necessary improvements to ensure a stable and reliable product.

\subsection{Proof of concept}
The team made several iterations of the project throughout its lifecycle. The first proof of concept product was made of a cardboard chess board, wires connected with electrical tape, most of the components not interacting with each other. This version of the product allowed
for rapid prototyping. The final product was much more refined, with a wooden chess board, embedded LEDs, and all of the components working together as one.
Despite the hardware not functioning at the live demo, the team has established the concept of integrating over-the-board and online play as a viable and desirable solution. This opens up opportunities for further development and refinement of the product.


\subsection{Team collaboration, time management, and problem-solving}
The team's ability to get the hardware working, even for a short period, and the software fully functional reflects effective collaboration and problem-solving skills. This bodes well for addressing the current issue with the hardware and for tackling future challenges in the project.
The team was able to effectively work together and accomplish a tremendous amount of work in the span of the course. We were able to focus on the main requirements of the project first to create the best final product that we could.

\section{Key Problem Areas}
 
Exploring the challenges encountered during the development of an innovative chess solution, we reflect on the factors that contributed to the hardware issues and examine the lessons learned to ensure future success in merging over-the-board and online chess experiences.

\subsection{Limitations of the product}
There were a few limitations of this product identified. The main ones being that the device always has to be plugged into the wall, and that is has to have a constant Internet connection to work properly. The physical size of the product is also 
quite large and is not very portable. All of these issues could be resolved with more money invested on components. However, they were not required to accomplish the main goals of the project.

\subsection{Insufficient testing and troubleshooting}
While there were known hardware issues before the live demo, it appears that these problems were not adequately addressed, leaving the hardware vulnerable to failure when presented in a live setting.
This was also the case with the software, where we did not adequately test it due to the dependency of it on the hardware, leading to inconsistent results.

\subsection{Limited contingency planning}
It seems there was no backup plan or alternative hardware setup available during the live demo. Having a contingency plan in place could have helped mitigate the impact of hardware failures for the live demo.

\subsection{Fragile connections}
The loose wires in the breadboard led to unstable connections and components, such as the LCD screen, ceasing to function. This further impacted the hardware's reliability and performance for the live demo.

\section{What Would you Do Differently Next Time}

In light of the challenges faced during the development of this project, here are some suggestions on what could be done differently next time to ensure a more successful outcome:

\subsection{Opt for a more robust hardware setup}
Replace breadboards with a more reliable solution, such as a custom-designed printed circuit board (PCB) or soldered connections, to minimize loose wires and improve the stability of the hardware components.

\subsection{Conduct thorough testing and troubleshooting}
Perform extensive tests and simulations under various conditions to identify potential hardware and software issues. Address these issues proactively to ensure a more reliable system during live demonstrations and real-world usage.

\subsection{Implement redundancy}
Introduce redundancy in critical hardware components, such as having a backup microcontroller or additional connections. This would help minimize the impact of a single component's failure on the overall system.

\subsection{Seek expert advice or mentorship}
Consult with experienced professionals or mentors who have successfully developed similar projects. Their insights can help guide the team in addressing challenges and identifying potential pitfalls early in the development process. \\ 

By adopting these strategies, the team can learn from the previous experience, improve the overall design and reliability of the project, and increase the likelihood of a successful outcome in future iterations.

\end{document}