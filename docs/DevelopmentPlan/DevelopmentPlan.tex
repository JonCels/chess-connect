\documentclass{article}

\usepackage{booktabs}
\usepackage{tabularx}

\input{../Comments}
%% Common Parts

\newcommand{\progname}{Chess Connect} % PUT YOUR PROGRAM NAME HERE
\newcommand{\authname}{Team \#4,
\\ Alexander Van Kralingen
\\ Arshdeep Aujla
\\ Jonathan Cels
\\ Joshua Chapman
\\ Rupinder Nagra} % AUTHOR NAMES without MacIDs 

\usepackage{hyperref}
    \hypersetup{colorlinks=true, linkcolor=blue, citecolor=blue, filecolor=blue,
                urlcolor=blue, unicode=false}
    \urlstyle{same}

\newcommand{\projectoverview}{

The Chess Connect project allows two users to play a game of chess on a physical board with the information being transmitted to an online web application over Bluetooth.
Currently, there is no way for players to seamlessly switch between playing on a physical board and playing online, but Chess Connect intends to change this by creating a central platform that will provide flexibility and remove barriers for new players looking to learn the game.

}

\title{Development Plan\\\progname}

\author{\authname}

\date{2022-09-26}

\begin{document}

\maketitle
\thispagestyle{empty}

~\newpage

\pagenumbering{roman}

\begin{table}[hp]
\caption{Revision History} \label{TblRevisionHistory}
\begin{tabularx}{\textwidth}{llX}
\toprule
\textbf{Date} & \textbf{Developer(s)} & \textbf{Change}\\
\midrule
2022-09-26 & Jonathan Cels & Team meeting and communication plans, personal role, workflow plan, coding standard, project scheduling, started technology section\\
2022-09-26 & Alexander Van Kralingen & Introduce CI/CD plan, team member role\\
2022-09-26 & Arshdeep Aujla & Introduce Hardware List 6, 3.2 Add Roles\\
2022-09-26 & Joshua Chapman & Modify Hardware List 6, 3.2 Add Role\\
2022-09-26 & Alexander Van Kralingen & Fix author field not being populated bug\\
2022-09-26 & Alexander Van Kralingen & Populate proof of concept plan\\
2022-09-26 & Rupinder Nagra & editing\\
\bottomrule
\end{tabularx}
\end{table}

\newpage

\maketitle

\section{Team Meeting Plan}
{The team will meet weekly on Thursdays at 10:20 AM until 11:20 AM. 
Team members are expected to attend the Thursday lecture and meet up following the lecture. 
On the event that no lecture is scheduled, the meeting time shall be changed to 9:30 AM until 11:20 AM.}

\medskip
\noindent{Additional meetings will be scheduled when necessary using the communication methods outlined in the Team Communication Plan.}

\section{Team Communication Plan}
{The team will communicate over a Discord group channel. Each team member is expected to have Discord downloaded and readily accessible on at least one of their devices. 
In the case of emergencies or time-critical situations, a team member will use SMS text messaging in place of Discord. Text messages will only be used in critical situations.}

\medskip
\noindent{Issues will be managed using a Kanban board, implemented with a Git project board. Tasks will be split into small, workable items with differing deadlines and importance.}

\section{Team Member Roles}
\subsection{Alexander Van Kralingen}
\begin{itemize}
    \item Embedded systems software development
    \item Continuous Integration/Deployment management
    \item Assisting with development of web application
    \item Code reviewing for both embedded and web application code
    \item Unit/integration testing development
\end{itemize}

\subsection{Arshdeep Aujla}
\begin{itemize}
    \item Assisting in building the chess board and customise the chess pieces
    \item Installing the hardware components on a breadboard
    \item Soldering components on the PCB
    \item Mapping the relation between the inputs and outputs using Karnaugh Map and State Machine Table 
\end{itemize}

\subsection{Jonathan Cels}
\begin{itemize}
    \item Enabling the Bluetooth connection between the application and the microcontroller
    \item Assisting with development of web application
    \item Programming the microcontroller with multiplexing, Bluetooth and rules of chess
    \item Leading testing initiatives and ensuring that thorough testing is completed
    \item Editing and formatting of all documentation before submission
\end{itemize}

\subsection{Joshua Chapman}
\begin{itemize}
    \item Circuit design of power and transmission of sensing units
    \item Hardware component selection and integration
    \item Assisting with soldering components and mechanical assembly
    \item Programming the microcontroller with multiplexing, Bluetooth and rules of chess
    \item Team leader for purchasing and planning phase
\end{itemize}

\subsection{Rupinder Nagra}
\begin{itemize}
    \item Lead front-end developer
    \item Deployment and testing of web application
    \item Back-end development for server/database management
    \item Code reviewing for all web application-related changes
\end{itemize}

\section{Workflow Plan}
{The project will use the feature-branch methodology in Git. An outline of the workflow is as follows:}
\begin{enumerate}
    \item Pull any changes from the master branch
    \item Create a new branch for development of a specific feature or subsystem
        \begin{enumerate}
            \item Use branch names that are descriptive to the feature
        \end{enumerate}
    \item Commit code frequently with descriptive messages
    \item Add unit and integration tests for the changes
    \item Push code to branch
    \item Create a pull request
    \item Another team member reviews and approves or rejects the pull request
        \begin{enumerate}
            \item The tests are reviewed and more tests are created if necessary
            \item The pull request cannot be approved if all tests do not pass
        \end{enumerate}
    \item Merge feature branch into the main branch
\end{enumerate}

\section{Proof of Concept Demonstration Plan}

The proof of concept demonstration will involve both a hardware component and a software component.
\paragraph{Hardware}
\begin{itemize}
    \item Components Involved: 
    \begin{itemize}
        \item Three different chess pieces with embedded magnets (ex: white queen, white pawn, black knight).
        \item Two 3D-printed squares with Hall-effect sensors and LEDs attached.
        \item LCD screen is configured and wired to display piece recognition. 
    \end{itemize}
    \item Demonstration:
    \begin{enumerate}
        \item Separate pieces are placed on the two squares.  
        \item One of the pieces will capture the other as per the rules of chess.
        \item The LCD screen will display the states of each square including the colour and type of piece. 
    \end{enumerate} 
    \item Risks:
    \begin{enumerate}
        \item The configuration of the sensors might be challenging to coordinate (sixty-four different sensors means many IOs to process at once). Correctly configuring two pieces means this can be easily scaled to sixty-four.
        \item The sensor accuracy is something that could create a barrier to identifying six different pieces in each colour. Inaccuracy could mean misreading the identity of the piece on the square. Correctly sensing three different pieces will prove that this barrier can be overcome.
        \item Supply-chain issues could mean some of the products selected for this project may not arrive on time for the demonstration.
    \end{enumerate}
\end{itemize}

\paragraph{Software}
\begin{itemize}
    \item Concepts to Demonstrate:
    \begin{itemize}
        \item Front-end User Interface (UI)
    \end{itemize}
    \item Demo Platform:
    \begin{itemize}
        \item \href{https://www.figma.com/}{Figma Design Platform}
    \end{itemize}
    \item Demonstration
    \begin{itemize}
        \item The UI layout will be shown, as well as all of the various components that will (eventually) connect to the back-end and hardware. The Bluetooth connection will provide the information needed to populate the UI with pieces, valid moves, best moves, game modes, etc. The Bluetooth connection will not be a part of this proof of concept because it is a tool that has been tested and proven in many other applications. 
    \end{itemize}
    \item Risks:
    \begin{enumerate}
        \item Hardware-software connection could be problematic due to signal loss or dropped communication packets. This will not be demonstrated in the proof of concept plan. However, Bluetooth is a commonly used communication method and should be a reliable information stream for the web application.
        \item Communication speed could limit the processing of valid moves and add delays to the connection between the moving pieces and the display on the UI. The speed in which pieces are identified and processed can show that this information can be sent over Bluetooth and processed for the web app quickly and efficiently.
        \item Chess engines could be resource intensive and the web application hosting platform and user system could prevent the user experience from being smooth and responsive. Displaying a light interface without too many calculations involved will minimize the computing power needed to provide a smooth user experience.
    \end{enumerate}
\end{itemize}

\section{Technology}

\begin{itemize}
\item Languages and Frameworks
\begin{itemize}
    \item \textbf{JavaScript, HTML, CSS, and React:} Front-end web based development
    \item \textbf{Python, FastAPI, Node.js:} Back-end development and Bluetooth connection
    \item \textbf{C:} Microcontroller programming
    \item \textbf{MongoDB, MySQL:} Database solution
\end{itemize}

\item Linting
\begin{itemize}
    \item \textbf{ESLint:} JavaScript development in VSCode
    \item \textbf{Flake8:} Python development in VSCode
\end{itemize} 

\item Testing Frameworks
\begin{itemize}
    \item \textbf{React Testing Library:} JavaScript testing framework for React
    \item \textbf{PyTest:} Python testing framework
\end{itemize}

\item \textbf{Heroku:} Deployment

\item \textbf{VSCode:} Code editor

\item \textbf{GitHub:} Version control and project management

\item \textbf{LaTeX:} Documentation

\item Libraries and API's
\begin{itemize}
    \item \textbf{Bluez, PyBluez:} Enabling Bluetooth data transfer
    \item \textbf{Stockfish.js:} Chess engine to find optimal moves
\end{itemize}

\item \textbf{Continuous Integration (CI)} GitHub Actions are to be used for CI, with the following general workflow:
\begin{itemize}
    \item Running on ubuntu-latest
    \item \textbf{Pull Requests (PR):} 
    \begin{itemize}
        \item Requires one reviewer
        \item Requires development branch to be up-to-date with main before approval
        \item Requires building/linting and tests to run completely without errors
    \end{itemize}
    \item \textbf{Build Embedded Code:}
    \begin{itemize}
        \item Triggered on PR from any hardware branches
        \item Checkout hardware branch
        \item Build all C code
        \item Run all C tests
    \end{itemize}
    \item \textbf{Build Web App Code:} 
    \begin{itemize}
        \item Triggered on PR from any webapp branches
        \item Checkout webapp branch
        \item Lint all Python code
        \item Lint all JavaScript code
        \item Build JavaScript code
        \item Run all JavaScript tests
        \item Run all Python tests
    \end{itemize}
    \item \textbf{Build LaTeX docs:} 
    \begin{itemize}
        \item Triggered on push (any branch) with *.tex file changes
        \item Build all latex documents
    \end{itemize}
    \item \textbf{Embedded Testing:} 
    \begin{itemize}
        \item Triggered on PR with Hardware Label opened, updated or closed
        \item Run all C tests
    \end{itemize}
    \item \textbf{Web-App Testing:} 
    \begin{itemize}
        \item Triggered on PR with Webapp Label opened, updated or closed
        \item Run all JavaScript tests
        \item Run all Python tests
    \end{itemize}

\end{itemize}
\item \textbf{Continuous Deployment (CD)} GitHub integration in Heroku to be used for continuous deployment:
\begin{itemize}
    \item \textbf{Deploy:}
    \begin{itemize}
        \item Triggered on push to main
        \item Push to main event occurs when merging a PR
        \item Web application deployed to Heroku on successfully merged PR.
    \end{itemize}
\end{itemize}

\item \textbf{Hardware}
\begin{itemize}
    \item \textbf{Design Tools}
    \begin{itemize}
        \item Altium for circuit design and testing
        \item AutoCAD for mechanical modeling and printing
    \end{itemize}
    \item \textbf{Chess Board:}
    \begin{itemize}
        \item Custom chess board with embedded circuit 
        \item Chess pieces with embedded magnets to facilitate piece detection
    \end{itemize}
    \item \textbf{Electrical Components}
    \begin{itemize}
        \item Hall sensor for piece detection
        \item RGB LEDs for move display
        \item LCD display on board for best engine move suggestions
        \item Mechanical switch component
    \end{itemize}
    \item \textbf{Processor:}
    \begin{itemize}
        \item Programmed for I/O handling and multiplexing
        \item Bluetooth unit to transmit custom packets back and forth
    \end{itemize}

\end{itemize}
\end{itemize}

\section{Coding Standard}
{The project will follow the \href{https://github.com/airbnb/javascript}{Airbnb style guide} for JavaScript development, and use the \href{https://flake8.pycqa.org/en/latest/}{Flake8 style guide} for Python development.}

\section{Project Scheduling}
{The project will use a GitHub project board to track and schedule tasks on a weekly basis.} 

\medskip
\noindent{Technical roles are decided on the basis of prior knowledge and interest. In the case of the team leader, the role will change with every deliverable. Team members are responsible for decomposing their tasks into kanban items, individually or alongside other team members working on the same task.}

\medskip
\noindent{Major milestones will be tracked on the project board. Milestones include both hard and soft deadlines for task completion. Hard deadlines are the project deliverable due dates. Soft deadlines are decided by the team for the completion of technical tasks.}
\end{document}